\documentclass[12pt]{report}
\usepackage[utf8]{inputenc}
\usepackage{setspace}
\usepackage{csquotes}
\usepackage{chicago}
\setlength{\parindent}{1.5em}
\setlength{\parskip}{0.5em}
\doublespacing

\begin{document}

\title{\textbf{Auditory Ethics: The Politics of Sound in Modern and Contemporary Prose}}
\author{[Author Name]}
\date{Ph.D. Thesis \\ \today}
\maketitle

\tableofcontents

\chapter*{Abstract}
\addcontentsline{toc}{chapter}{Abstract}
This thesis investigates the intersection of sound and ethics in twentieth- and twenty-first-century prose, examining how modern and contemporary authors incorporate auditory experience into narrative form and content. It expands the scope of analysis beyond canonical figures such as James Joyce, Virginia Woolf, Toni Morrison, and Maxine Hong Kingston to include a wider range of modernist, postcolonial, and postmodern writers. Drawing on interdisciplinary theoretical perspectives from sound studies, narrative ethics, postcolonial theory, and literary modernism/postmodernism, the project develops the concept of “auditory ethics” to describe how literature engages the politics of sound—who is heard, who is silenced, and how acts of listening are represented. The methodology combines close reading with corpus linguistics and computational text analysis, using digital tools to complement qualitative interpretation. Through case studies spanning high modernist experiments with interior monologue, postcolonial narratives reclaiming oral traditions, African American literature’s engagement with music and voice, and postmodern fiction’s confrontation with noise and media, the thesis demonstrates that sound in literature is never neutral. Rather, literary sounds and voices carry ethical weight and political implications, challenging readers to become active listeners. The study finds that authors across diverse contexts use auditory techniques to bridge distances between characters, critique power dynamics, and invite empathetic engagement. By foregrounding the auditory dimension of prose, these writers transform reading into an ethical act of listening. The thesis concludes that attending to the “auditory ethics” of literature enriches our understanding of narrative form and its social resonances. An extensive bibliography is provided in Chicago style. 

\chapter{Introduction}
Sound is fundamental to human experience, yet literature—an ostensibly silent, written medium—has continually sought to evoke and harness the auditory realm. This dissertation examines how modern and contemporary prose writers use sound as a narrative and ethical resource. Titled \textit{“Auditory Ethics: The Politics of Sound in Modern and Contemporary Prose,”} the study explores the intersection of auditory experience with questions of ethics and power in fiction from roughly the early twentieth century to the present. It asks: In what ways do novelists and storytellers incorporate the senses of hearing and sound into their works, and to what effect? How do literary representations of sound (from spoken voices and dialogue to musical structures, noise, silence, and the rhythms of language itself) engage with ethical questions about empathy, otherness, and justice? And how do these auditory elements reflect or challenge the politics of who gets to speak and who is heard in society?

The phrase “auditory ethics” in the title signifies the dissertation’s central claim: that paying attention to sound in literature—both the sounds depicted within stories and the sound-like qualities of the narrative voice or language—reveals an ethical dimension of storytelling. Many writers implicitly treat listening as an ethical act, one that can bridge distances between people or highlight injustices when certain voices are ignored. The “politics of sound” refers to how power relations play out through sound: whose voices dominate or are marginalized, how noise can be used to oppress or to resist, and how modern technological and cultural shifts in the auditory domain influence literature. By analyzing prose through this lens, the study uncovers the strategies authors use to politically charge the act of listening and the representation of sound.

This research is rooted in literary analysis but draws upon multiple theoretical frameworks. It engages with \textbf{sound studies}, an interdisciplinary field that examines sound’s cultural, technological, and philosophical aspects. Insights from sound studies help illuminate how literature responds to what has been called the “sonic turn” in modern culture, when advancements like the phonograph, radio, and telephone changed how people experienced sound [oai_citation_attribution:0‡uknowledge.uky.edu](https://uknowledge.uky.edu/english_etds/29/#:~:text=acts%20of%20listening%20in%20modern,perceiving%20self%20in%20increasingly%20urban) [oai_citation_attribution:1‡floridapress.blog](https://floridapress.blog/2018/12/04/modernist-soundscapes/#:~:text=She%20argues%20that%20the%20common,reproduction%20of%20the%20tape%20recorder). The project also employs concepts from \textbf{narrative ethics}, which is concerned with the moral relations between storytellers, characters, and readers. Narrative ethics provides tools to discuss how giving a character a voice or asking the reader to “listen” to a text can be seen as ethical or unethical. In addition, the thesis incorporates \textbf{postcolonial theory} to address literature emerging from colonial and formerly colonized contexts, where issues of voice, silencing, and the recovery of oral traditions are paramount. Finally, it situates its discussion in the literary-historical contexts of \textbf{modernism and postmodernism}, two movements that, in different ways, revolutionized artistic form—including experiments with sound and language in prose.

A key goal of this study is to avoid privileging any single theoretical framework at the expense of others. Instead, it demonstrates how these perspectives can complement one another. For example, a close reading of a novel by \textit{Virginia Woolf} can be enriched by sound studies (noting her use of onomatopoeia and rhythm), by narrative ethics (considering how her narrative technique invites empathetic listening to characters’ inner lives), and by a postcolonial awareness (if examining how voices of different social backgrounds appear in her work). In treating these frameworks even-handedly, the thesis shows that the politics of sound in literature is a multifaceted subject: it is at once formal/aesthetic, ethical, and sociopolitical.

The corpus of literature examined is purposefully broad. While the initial inspiration comes from authors mentioned in the abstract—figures like James Joyce, Woolf, Toni Morrison, and Maxine Hong Kingston, all of whom explicitly concern themselves with voice and sound—this study extends to additional writers across the anglophone world. High modernist authors such as \textit{Dorothy Richardson}, \textit{William Faulkner}, and \textit{Samuel Beckett} feature in the discussion alongside Joyce and Woolf, illustrating how early twentieth-century prose experimented with auditory techniques. Postcolonial and transnational voices including \textit{Chinua Achebe}, \textit{Ngũgĩ wa Thiong’o}, and \textit{Salman Rushdie}, as well as Asian American and diasporic writers like Kingston, broaden the scope to colonial and immigrant contexts where reclaiming voice is often a central theme. African American literature receives dedicated attention through the works of \textit{Ralph Ellison}, \textit{Zora Neale Hurston}, and Morrison, highlighting the legacy of oral culture and music (from folktales to jazz and blues) in shaping narrative form and ethics. Finally, late twentieth-century and postmodern writers such as \textit{Don DeLillo} and \textit{Thomas Pynchon} are considered to show how the theme of sound (and conversely, noise) evolves in an era of information saturation and media networks. By examining this wide array of authors, the thesis underscores that the role of sound in literature is not confined to any one period or genre—it is a pervasive and evolving concern.

Methodologically, this project combines traditional literary scholarship with digital humanities approaches. Close reading remains the primary mode of analysis: careful attention is given to how specific texts represent sound and listening. For instance, what words, images, or narrative structures do authors use to evoke sound? How do they describe acts of listening or the absence of sound (silence)? How is dialogue rendered on the page, and does it have qualities that suggest particular accents, rhythms, or musicality? Such questions are addressed through detailed textual analysis in the case study chapters. Alongside this, the thesis employs \textbf{corpus linguistics and computational text analysis} as a supplementary approach. Using a curated digital corpus of literary texts, the study quantifies certain patterns—for example, the frequency of sound-related words or onomatopoeic expressions in modernist vs. Victorian novels, or the prevalence of phrases like “listen” and “hear” in narratives of different eras. These quantitative findings provide empirical support and context for the interpretive claims, although they remain subservient to the qualitative analysis. In effect, the digital methods function as a magnifying lens, bringing into focus broad trends that might escape notice in isolated readings. For example, a computational scan can reveal that authors like Joyce and Morrison use an unusually high density of auditory imagery compared to their contemporaries, reinforcing the notion that sound is central to their art. Such data, when used carefully, enriches the argument without overshadowing the nuanced reading of individual works.

The structure of the dissertation reflects a conventional progression from theory and context to application and synthesis. Chapter 2 offers a review of relevant literature and theoretical background, surveying work in sound studies, narrative theory, postcolonial studies, and related fields to establish the scholarly landscape. Chapter 3 discusses the methodology in detail, explaining how close reading and computational analysis are conducted and justified. Chapter 4 presents a series of case studies grouped by theme and historical context, each examining selected authors and texts to illuminate different facets of auditory ethics: from modernist “soundscapes” to postcolonial orature, from the sonic ethos of African American storytelling to the metafictional play with noise in postmodernism. Chapter 5 provides an analysis and discussion that draws connections among the case studies, synthesizing findings and linking back to the theoretical frameworks. Finally, Chapter 6 concludes the study, summarizing the insights gained and reflecting on their implications for literary criticism and our understanding of narrative ethics. An extensive bibliography follows, compiled in Chicago style, documenting primary sources (the literary works) and secondary sources from the diverse theoretical and critical literature that inform this project.

In summary, the introduction has situated the research problem and objectives of this thesis. Literature is often thought of as a visual medium (print on a page) intended for silent reading, yet this study argues that the sonic dimension of prose is both rich and consequential. Modern and contemporary authors have woven sound into their narratives in innovative ways, and these auditory elements carry ethical and political significance—whether it be giving voice to the voiceless, using musical form to suggest harmony or dissonance in society, or foregrounding noise to critique modern life. By traversing multiple literary periods and theoretical approaches, the chapters that follow will demonstrate that exploring the “auditory ethics” of prose offers fresh perspectives on familiar texts and opens up new conversations about the role of the senses in literature. The ultimate hope is that readers of this thesis will come away with a heightened awareness of how listening, as much as seeing or reading, is integral to the experience of narrative, and with an appreciation for the many ways authors use sound to make meaning and pose moral questions.

\chapter{Literature Review}
\section{Sound, Modernity, and the Literary Soundscape}
The field of \textbf{sound studies} provides a crucial starting point for understanding how literature engages with sound. Sound studies scholars have noted that Western culture has long been dominated by what is called “ocularcentrism,” a prioritization of sight over other senses [oai_citation_attribution:2‡floridapress.blog](https://floridapress.blog/2018/12/04/modernist-soundscapes/#:~:text=These%20writers%20challenged%20ocularcentrism%2C%20the,the%20course%20of%20contemporary%20literature). In contrast, the late nineteenth and early twentieth centuries saw the emergence of new audio technologies and a growing interest in auditory experience, often termed the “sonic modernity” of the era [oai_citation_attribution:3‡journals.openedition.org](https://journals.openedition.org/ebc/2322#:~:text=Sam%20Halliday%2C%20Sonic%20Modernity%3A%20Representing,Literature%2C%20Culture%20and%20the%20Arts) [oai_citation_attribution:4‡journals.openedition.org](https://journals.openedition.org/ebc/2322#:~:text=1%20Sam%20Halliday%E2%80%99s%20dashingly%20titled,%E2%80%A6%5D%20is%20not%20historically%20specific%E2%80%99). Canadian composer R. Murray Schafer’s influential work on soundscapes introduced the idea that every environment has its own ensemble of sounds, and he famously urged a rethinking of our “soundscape” to counter the unchecked rise of noise pollution (what he dubbed the “sound sewer” of modern life) [oai_citation_attribution:5‡acousticbulletin.com](https://www.acousticbulletin.com/our-visual-focus-part-2-the-eye-versus-the-ear/#:~:text=Bulletin%20www,that%20willingly%20trades%20its). Schafer also treated literary descriptions as valuable records of historical soundscapes, calling authors “reliable ‘earwitnesses’” to how the world once sounded [oai_citation_attribution:6‡jcls.io](https://jcls.io/article/id/3583/#:~:text=example%2C%20in%20Schafer%20,9%20%20in%20%2033). This concept of the writer as an “earwitness” underscores that novelists and poets have actively documented and artistically reshaped the auditory textures of their times, not just its visual details.

A number of studies specifically examine the representation of sound in literature, coalescing into what has been termed \textbf{literary acoustics} or the study of the literary soundscape. Sam Halliday’s \textit{Sonic Modernity} (2013), for example, explores how early twentieth-century British and American literature reacted to modern sound phenomena—from urban noise to new music—arguing that “sound is finally receiving the sort of attention the history of aesthetics has traditionally reserved for the image” [oai_citation_attribution:7‡journals.openedition.org](https://journals.openedition.org/ebc/2322#:~:text=live%20modernity%21%E2%80%9D%E2%80%99). Angela Frattarola’s \textit{Modernist Soundscapes: Auditory Technology and the Novel} (2018) likewise examines how technologies like the phonograph and radio influenced modernist writers, enabling new narrative strategies. Frattarola notes that devices such as headphones, which could “pipe sounds from afar into a listener’s headspace,” inspired modernists to experiment with recording the interior monologues of characters in a stream-of-consciousness style [oai_citation_attribution:8‡floridapress.blog](https://floridapress.blog/2018/12/04/modernist-soundscapes/#:~:text=She%20argues%20that%20the%20common,reproduction%20of%20the%20tape%20recorder). Authors like James Joyce and Virginia Woolf, she argues, effectively attempted to transcribe the mind’s “inner sound,” blurring the boundary between external and internal auditory experience [oai_citation_attribution:9‡floridapress.blog](https://floridapress.blog/2018/12/04/modernist-soundscapes/#:~:text=She%20argues%20that%20the%20common,reproduction%20of%20the%20tape%20recorder). Indeed, Woolf’s celebrated use of onomatopoeia can be seen as an effort to capture “the sounds of the world without mediation” on the page [oai_citation_attribution:10‡floridapress.blog](https://floridapress.blog/2018/12/04/modernist-soundscapes/#:~:text=a%20listener%E2%80%99s%20headspace%2C%20inspired%20modernists,reproduction%20of%20the%20tape%20recorder). In Joyce’s works, especially the “Sirens” episode of \textit{Ulysses}, language is pushed to mimic music and ambient noise; Joyce “made words represent music by playing with or even overcoming certain conventional features of language” [oai_citation_attribution:11‡bibliolore.org](https://bibliolore.org/2022/02/02/joyces-musical-sirens/#:~:text=In%20the%20%E2%80%9CSirens%E2%80%9D%20episode%20of,certain%20conventional%20features%20of%20language). These studies situate literary techniques in a wider context of sonic modernity, suggesting that narrative innovations often paralleled contemporary shifts in how people listened and what they heard in an age of cacophony and mechanically reproduced sound.

Another relevant concept is the idea of \textbf{audionarratology}, which merges narratology (the study of narrative structure) with sound studies. Audionarratology examines how narratives can evoke sound and how the act of listening can be thematized in texts. As one scholar puts it, reading a literary text can trigger “phonological recoding” – an inner voice in the reader’s mind – meaning that even silent reading has an implicit auditory component [oai_citation_attribution:12‡jcls.io](https://jcls.io/article/id/3583/#:~:text=recitation%20of%20literary%20texts%2C%20where,of%20reading%20whether%20aloud%20or). From this perspective, devices like alliteration, assonance, and rhythm are not mere ornaments but integral to how a story communicates, effectively creating a “sound design” that the reader experiences internally [oai_citation_attribution:13‡jcls.io](https://jcls.io/article/id/3583/#:~:text=approach%20to%20sound%20in%20literature,31). Moreover, audionarratologists study how texts represent listening within the story world. For example, characters may be depicted as keenly listening to their environment, or entire plots might revolve around sounds (a mysterious noise, a spoken secret, a radio broadcast). These elements contribute to what has been called the “theatre of the mind” in narrative, wherein described sounds conjure vivid imaginative experiences for the reader [oai_citation_attribution:14‡jcls.io](https://jcls.io/article/id/3583/#:~:text=There%20have%20been%20numerous%20recent,33%20Snaith). Recent collections such as \textit{Literature and Sound} (2020, edited by Anna Snaith) and \textit{Audionarratology} (2016, edited by J. Mildorf and T. Kinzel) bring together such analyses, covering topics from the role of musical structures in narrative to the function of silence in fiction. One finding from this body of work is that certain genres historically place more emphasis on ambient sound description—Gothic fiction, for instance, often foregrounds wind, whispers, and other sounds to create suspense [oai_citation_attribution:15‡jcls.io](https://jcls.io/article/id/3583/#:~:text=There%20have%20been%20numerous%20recent,33%20Snaith). In contrast, realist literature might relegate sound to the background unless it directly advances the plot (such as a character overhearing a crucial conversation). This thesis builds on audionarratological insights by investigating how attentiveness to sound in narrative correlates with ethical themes, such as empathy and otherness.

Perhaps the most directly relevant aspect of sound studies to this thesis is the exploration of \textbf{sound and power}, especially the distinction between sound that is meaningful (like speech or music) and “noise” that is unwanted or disruptive. French theorist Jacques Attali’s oft-cited work \textit{Noise: The Political Economy of Music} (1977, English trans. 1985) contends that noise is not just a sonic phenomenon but a social one: “the mere naming of something as noise is a political action” in that it signifies a threat to the established order [oai_citation_attribution:16‡musicandsoundstudies.wordpress.com](https://musicandsoundstudies.wordpress.com/2014/03/03/noise-pt-2-attali/#:~:text=This%20noise%20%E2%80%94%20the%20voice,As%20Attali) [oai_citation_attribution:17‡musicandsoundstudies.wordpress.com](https://musicandsoundstudies.wordpress.com/2014/03/03/noise-pt-2-attali/#:~:text=boundaries%2C%20it%20literally%20,As%20he%20concisely). Attali argues that organized sound (music) has historically been used to impose order—he metaphorically describes music as a ritualized “sacrifice” that creates harmony by excluding dissonance [oai_citation_attribution:18‡musicandsoundstudies.wordpress.com](https://musicandsoundstudies.wordpress.com/2014/03/03/noise-pt-2-attali/#:~:text=To%20explain%20music%20as%20the,audible%20to%20the%20perceptive%20ear). Conversely, noise represents whatever is culturally labeled as disordered or undesirable sound, often the voice of the marginalized or the chaotic din of rebellion. “This noise — the voice of dissent and disorder — threatens to uproot the existing social order,” Attali writes, and therefore societies seek to contain it [oai_citation_attribution:19‡musicandsoundstudies.wordpress.com](https://musicandsoundstudies.wordpress.com/2014/03/03/noise-pt-2-attali/#:~:text=This%20noise%20%E2%80%94%20the%20voice,As%20Attali). His bold claim that “music is prophecy” encapsulates the idea that by listening to what is considered noise, one can hear the rumblings of societal change [oai_citation_attribution:20‡musicandsoundstudies.wordpress.com](https://musicandsoundstudies.wordpress.com/2014/03/03/noise-pt-2-attali/#:~:text=This%20noise%20%E2%80%94%20the%20voice,As%20Attali). In literary terms, we can apply Attali’s insight by asking: which characters or cultural elements are treated as “noise” within a narrative? For example, a novel might portray the songs of a subaltern people as mere background noise in the ears of colonizers, reflecting the latter’s power to dismiss those voices. On the other hand, a writer might stylistically elevate such “noise” into music or poetry on the page, as an act of resistance. Jennifer Stoever’s concept of the “sonic color line” extends the discussion of sound and power into the realm of race. In \textit{The Sonic Color Line: Race and the Cultural Politics of Listening} (2016), Stoever examines how in the United States, listening practices were racialized – how certain voices or sounds were coded as “black” or “white” and valorized or stigmatized accordingly [oai_citation_attribution:21‡aaihs.org](https://www.aaihs.org/the-sonic-color-line-black-women-and-police-violence/#:~:text=aural%20border%20between%20white%20people,call%20the%20sonic%20color%20line). She defines the sonic color line as “the learned cultural mechanism that establishes racial difference through listening habits and uses sound to communicate one’s position vis-à-vis white citizenship” [oai_citation_attribution:22‡aaihs.org](https://www.aaihs.org/the-sonic-color-line-black-women-and-police-violence/#:~:text=aural%20border%20between%20white%20people,call%20the%20sonic%20color%20line). For instance, African American music might be appropriated as entertainment yet the actual voices of black people in public (their cries of protest or grief) might be ignored or labeled as noise [oai_citation_attribution:23‡aaihs.org](https://www.aaihs.org/the-sonic-color-line-black-women-and-police-violence/#:~:text=presence%20%20of%20a%20long,call%20the%20sonic%20color%20line) [oai_citation_attribution:24‡aaihs.org](https://www.aaihs.org/the-sonic-color-line-black-women-and-police-violence/#:~:text=The%20patroller%E2%80%99s%20deliberate%20tone%20ensures,European%20musical%20concepts%20of%20the). Stoever’s framework will be particularly relevant when discussing African American literature (Section 4.3), but it also points to a broader methodological approach: in analyzing literary texts, one should pay attention to whose sounds are described as harmonious or meaningful and whose are depicted as cacophonous or negligible. Such narrative choices often mirror societal attitudes and thus carry ethical weight.

In sum, sound studies and related scholarship furnish this thesis with key concepts: the critique of ocularcentrism (and a corresponding validation of auditory ways of knowing) [oai_citation_attribution:25‡floridapress.blog](https://floridapress.blog/2018/12/04/modernist-soundscapes/#:~:text=These%20writers%20challenged%20ocularcentrism%2C%20the,the%20course%20of%20contemporary%20literature); the idea of literature as a repository and artistic transformation of historical soundscapes [oai_citation_attribution:26‡jcls.io](https://jcls.io/article/id/3583/#:~:text=sonic%20environments%20as%20he%20did,32%20in%20Snaith%202020%2C%2020); the narrative techniques that evoke sound and require an imagined listening by the reader; and the entanglement of sound with structures of power (who gets to speak, who is “noise”). As we move forward, these ideas will inform readings of specific literary works. Modernist literature’s fascination with sound, for example, can be seen as part of a larger cultural moment that challenged the primacy of visual perception and linear logic, embracing instead the “immediate and unifying” nature of listening [oai_citation_attribution:27‡floridapress.blog](https://floridapress.blog/2018/12/04/modernist-soundscapes/#:~:text=These%20writers%20challenged%20ocularcentrism%2C%20the,the%20course%20of%20contemporary%20literature). Likewise, the attention to dialect and oral tradition in postcolonial writing resonates with the sound-studies recognition that voice and listening are central to identity and agency. The literature review now turns to narrative theory and ethics, and postcolonial theory, to develop those complementary angles.

\section{Narrative Voice and Ethics of Listening}
The field of \textbf{narrative ethics} is concerned with the moral implications of storytelling practices. It asks how narratives position readers in relation to characters and to the truth of the story, and what responsibilities and power dynamics are involved in the act of narration. One foundational idea in narrative ethics (influenced by the philosophy of Emmanuel Levinas and others) is that encountering the “Other” – an entity outside oneself – is an ethical experience. In literature, this encounter often takes the form of a reader meeting a character or narrator through the medium of text. Scholar Adam Zachary Newton, in his book \textit{Narrative Ethics} (1995), posits that narrative itself is an event of ethical significance because it involves an interaction between self and other: author and reader, narrator and character, text and audience. Rather than viewing stories as closed worlds unto themselves, Newton suggests we consider the transactional nature of reading – essentially a form of listening. We “listen” to the voices of narrators and characters, and this act can mirror real-life ethical situations, such as listening to another person’s testimony or plea.

\textbf{Voice} in narrative theory is a multifaceted concept, but centrally it refers to the perspective and agency behind the words on the page. Who is speaking, and from what position? Is the narrative voice omniscient and authoritative, or limited and perhaps unreliable? Is the character allowed to speak in their own voice (through first-person narration or interior monologue), or is their story told by someone else? These questions have ethical dimensions. Wayne C. Booth, in \textit{The Company We Keep: An Ethics of Fiction} (1988), argued that narratives implicitly ask readers to trust certain voices and values; the “company” a reader keeps by believing or sympathizing with a narrator is an ethical choice shaped by the text. If a novel silences a character or presents them only through another’s judgment, it raises ethical concerns about misrepresentation or marginalization within the story’s world. By contrast, a novel that features multiple voices in dialogue (sometimes literally speaking in the text, sometimes through shifts in narrative point of view) can enact a kind of ethical pluralism, acknowledging no single perspective has a monopoly on truth.

Here we can invoke Mikhail Bakhtin’s influential concept of \textbf{polyphony} in the novel. In his study of Dostoevsky, Bakhtin praised the author’s ability to create a “polyphonic” narrative in which characters’ voices are not subordinated to a single authoritative narrator, but rather coexist and interact on relatively autonomous terms. Such a novel is “multi-voiced” and resists reducing characters to mere mouthpieces for the author’s viewpoint. Polyphony in this sense aligns with an ethical stance: it honors the distinctiveness of each voice, much as an ethical society would honor the autonomy and perspective of each individual. Bakhtin’s work was not explicitly ethical in a prescriptive sense, but later scholars have noted the ethical resonance of his ideas: a truly dialogic work is one that models listening and responsiveness among voices.

If we bring these narrative theories into conversation with sound, an intriguing parallel emerges: \textbf{reading as listening}. When an author crafts a first-person narrative or interior monologue, the reader is essentially placed in the position of a listener, overhearing the character’s voice. For instance, when we read the soliloquies in Faulkner’s \textit{The Sound and the Fury} or the interior monologue of Molly Bloom at the end of Joyce’s \textit{Ulysses}, we are asked to “listen in” on very intimate, unfiltered speech. This can create empathy, but it can also be demanding or uncomfortable—such narratives often lack the guiding commentary of an outside narrator, meaning we must make sense of the voice on its own terms. Philosopher and literary critic Martha Nussbaum has argued that this kind of engagement with a character’s inner voice can expand the reader’s moral imagination, training us in patience, attention, and empathy (as she discusses in works like \textit{Love’s Knowledge}, 1990). From the perspective of narrative ethics, the formal choice to grant a character their own voice is an ethical one: it invites the reader to treat that character as a subject with their own worldview, rather than as an object to be spoken about. In the context of this thesis, we will see authors like Toni Morrison and Maxine Hong Kingston using narrative structure to ensure that marginalized characters (African American women, Chinese American immigrants, etc.) have a voice that the reader must confront and engage with directly.

Another aspect of narrative voice is \textbf{reliability} and trust. A narrator who openly muses, hesitates, or self-corrects might be seen as inviting the reader’s judgment—are they telling the truth? do they see the whole picture? Unreliable narrators (such as the unnamed but clearly biased narrator of Ralph Ellison’s \textit{Invisible Man} during the famous “battle royal” scene, or the child narrator in some of Kingston’s tales who doesn’t fully understand what she hears) force readers to listen between the lines and consider what is not being said. This too is an ethical exercise: the reader must not passively accept a singular voice, but rather become an active, and at times skeptical, listener. The politics of sound enter here when we consider why a narrator might be unreliable or limited – often it is because they are constrained by their social context. In Ellison’s novel, for example, the African American narrator’s perspective is limited not by any personal failing but by the fact that society refuses to “see” or properly hear him (hence he is “invisible”). His first-person voice is passionate and vivid, but part of the novel’s strategy is to reveal to the reader how many voices like his go unheard in the real world. Thus, the reader is ethically implicated: having heard his story, what will we do with it? This aligns with what some narrative ethicists describe as the transfer of ethical responsibility to the reader once a story is told.

Within narrative theory, the \textbf{act of listening} itself has been thematized by some scholars. James Phelan and Peter Rabinowitz, in their work on narrative communication, use a model where authors “tell” stories to readers, but readers must also be willing to listen (or read receptively). The ethics of reading, then, involves a willingness to be responsive to a text’s voice. One might draw on the metaphor of an ethical listener: just as in interpersonal ethics, listening to another person with attention and openness is a virtue, similarly, reading a narrative empathetically can be seen as practicing an ethical form of listening. Of course, this can be taken too far—one should not accept everything a text implies uncritically, just as one might question a speaker who might be deceitful or biased. Thus, a balance must be struck between empathetic listening and critical analysis.

In sum, narrative theory contributes to our understanding of auditory ethics by illuminating how \textbf{voice} is constructed in fiction and why it matters. A story can either encourage us to listen to its characters or can drown them out with an overbearing narrative authority. Many of the works this thesis examines are notable for their polyvocality or for elevating voices that literary tradition had often sidelined (like those speaking in dialect or non-standard English). The “politics of sound” here intersects with the politics of narration: whose voice drives the narrative and whose voices are included as part of the story’s fabric. When we later analyze, for example, Zora Neale Hurston’s use of vernacular dialogue or Kingston’s blending of memoir with transcribed oral legends, we will see narrative ethics in action. These writers make deliberate stylistic choices to validate particular ways of speaking and to make readers hear them. In doing so, they implicitly argue for the value of those voices in the moral imagination of their audience.

\section{Postcolonial Perspectives: Voice, Orality, and Power}
Postcolonial theory offers another essential framework for this study, focusing on how literature from formerly colonized or marginalized cultures reclaims voice and agency. A key concern of postcolonial analysis is encapsulated in Gayatri Chakravorty Spivak’s famous question: “Can the subaltern speak?” [oai_citation_attribution:28‡forsea.co](https://forsea.co/silent-voices-the-subaltern-can-speak-have-always-spoken/#:~:text=%E2%80%9CCan%20the%20subaltern%20speak%3F%E2%80%9D%2C%20postcolonial,and%20always%20will%20speak). By “subaltern,” Spivak refers to those of inferior rank in colonial and class hierarchies – the colonized, the oppressed, particularly subaltern women – whose voices have been historically silenced or filtered through colonial discourse. Spivak’s answer was pessimistic: the subaltern’s voice is often so overwritten by dominant narratives that, in the strictest sense, it cannot be heard authentically by the centers of power. This critique has galvanized postcolonial writers and scholars to find ways to let the subaltern speak, or to highlight the structural reasons why that speech is not heard.

One of the most direct ways literature engages with this issue is through \textbf{orality} – integrating oral traditions, spoken language patterns, and storytelling techniques from non-Western or indigenous cultures into written texts. In many colonized societies, oral literature (folk tales, epics, songs, proverbs) was the primary medium of cultural transmission. Colonial powers often disparaged orality as a sign of backwardness, elevating the written word (especially when written in the colonizer’s language) as the only legitimate culture. Thus, when postcolonial or anticolonial writers incorporate oral elements into novels or memoirs, they are performing a double act: artistically enriching their narratives and asserting the worth of their heritage. There is an ethical and political assertion in something as simple as writing dialogue in a local dialect or transcribing a folk song into the text. It says: these sounds, these ways of speaking, carry knowledge and value. 

For example, Nigerian author Chinua Achebe’s groundbreaking novel \textit{Things Fall Apart} (1958) opens with an Igbo proverb: “Among the Igbo the art of conversation is regarded very highly, and proverbs are the palm-oil with which words are eaten.” [oai_citation_attribution:29‡impactnetwork.org](https://www.impactnetwork.org/latest-news/proverbs-are-the-palm-oil-with-which-words-are-eaten#:~:text=I%20was%20looking%20back%20at,oil%20with%20which%20words%20are). By including this line – and indeed dozens of Igbo proverbs throughout the novel – Achebe not only lends authenticity to the portrayal of Igbo life, but he also communicates a cultural philosophy that elevates skilled speech and listening to the level of a fine art. The proverb “proverbs are the palm-oil with which words are eaten” suggests that language, adorned with the wisdom of proverbs, is as vital and nourishing as food [oai_citation_attribution:30‡impactnetwork.org](https://www.impactnetwork.org/latest-news/proverbs-are-the-palm-oil-with-which-words-are-eaten#:~:text=I%20was%20looking%20back%20at,oil%20with%20which%20words%20are). It implicitly invites the reader (especially a Western reader unfamiliar with Igbo culture) to appreciate an aesthetics and ethics of orality: speaking well and listening to proverbial wisdom are central to Igbo ethics. In a colonial context where African voices were often dismissed, Achebe’s use of Igbo oral tradition in an English-language novel was a radical re-centering of narrative authority.

Kenyan writer Ngũgĩ wa Thiong’o took this a step further in both theory and practice. In his essay collection \textit{Decolonising the Mind} (1986), Ngũgĩ argues that language is a carrier of culture and that the domination of colonized peoples’ minds was achieved through the imposition of the colonizer’s language (like English) at the expense of native languages. He advocates a return to indigenous languages and oral forms as a means of cultural and mental liberation. Famously, Ngũgĩ decided to write his later novels in his mother tongue Gikuyu and then supervise their translation into English, rather than write in English directly. His novel \textit{Devil on the Cross} (1980, Gikuyu) exemplifies the integration of oral performance into print: it is framed as a story told by a traditional storyteller (a \textit{gĩcaandĩ} player) to a live audience. The narrative repeatedly reminds the reader that this is an oral tale being performed and heard. By doing so, Ngũgĩ collapses the distance between written literature and oral storytelling, insisting that the voices of Gikuyu peasants and workers he portrays are as important and deserving of attention as any formal written narrative. In terms of auditory ethics, Ngũgĩ’s work demands that the reader adopt the stance of an audience at an oral performance – to listen, to imagine the cadence and emotion behind the words, and thus to engage with the story on a more communal and participatory level than the typically solitary act of silent reading might encourage.

Another dimension of postcolonial sound politics is the use of \textbf{hybridity} in language. Authors like Salman Rushdie and Arundhati Roy infuse English prose with the sounds of South Asian speech—through code-switching (mixing languages), phonetic spellings that mimic accents, and rhythmic syntax that echoes oral narration. Rushdie, in \textit{Midnight’s Children} (1981), creates a narrator, Saleem Sinai, who addresses the reader (and an implied listener named Padma) directly, in a rambling, digressive style reminiscent of an oral storyteller trying to get a long story out before he “cracks under the burden of it.” The narrative is full of Indian idioms, Hindi-Urdu words, and even onomatopoeic representations of Indian English pronunciation. This linguistic soundscape has a political edge: it decenters Queen’s English and validates Indian Englishes as capable of bearing the weight of serious history and fiction. Moreover, Saleem’s narrative is itself a “voice” speaking back to official histories of India’s independence and partition, offering a counter-narrative that is personal, fantastical, and unabashedly noisy with the clamor of many voices (he is connected, via a fantastical radio-like telepathy, to all other children born at the moment of India’s independence, symbolizing the multitude of new voices of the nation). Rushdie’s style, which some have called “auditory exuberance,” illustrates how postcolonial prose often turns reading into an exercise of listening for voices between the lines of imperial history.

In postcolonial contexts, the line between sound and silence can be especially fraught. Colonial regimes often attempted to silence indigenous languages (for instance, through schools that punished students for speaking their mother tongue). In literature, this history appears in themes of enforced silence or voicelessness. Maxine Hong Kingston’s memoir-novel \textit{The Woman Warrior} (1976) begins with the command “You must not tell anyone,” referring to a silencing of women’s stories in her Chinese American family. Kingston’s text then becomes an imaginative act of unsilencing—she recounts (and to some extent reinvents) the story of a shamed aunt, tales of female warriors and ghosts, and her own struggles to speak up in American society. The book stylistically oscillates between actual remembered dialogue (sometimes in Cantonese, sometimes broken English) and mythic storytelling, reflecting Kingston’s bicultural oral heritage. The “sound” of \textit{The Woman Warrior} is the sound of a voice finding itself – sometimes halting, sometimes lyrical, mixing languages and registers. In giving form to what was culturally silenced, Kingston’s work exemplifies the auditory ethic of breaking silence, a common trope in postcolonial and minority literature. The politics here are clear: to speak (or to write a voice that speaks) is to exist and to resist erasure.

Finally, we should note how \textbf{music and performance} are used in postcolonial prose as extensions of oral culture. In many African, Caribbean, and Black diaspora narratives, music stands in for a collective voice or history that isn’t recorded in written archives. Toni Morrison’s novels (to foreshadow Section 4.3) often invoke slave songs, jazz, and blues as carriers of cultural memory. In Caribbean literature, as in the works of authors like Wilson Harris or Earl Lovelace, calypso or reggae rhythms can permeate the prose. These musical elements are not just stylistic flourishes; they represent forms of expression that slavery and colonialism could not destroy. The inclusion of music and oral performance in written narratives serves to remind readers that the text is part of a larger continuum of storytelling that transcends the page. It is also a political statement about survival and continuity of culture through sound.

In conclusion, postcolonial theory and literature underscore that sound—whether in the form of language, oral storytelling, or music—is deeply entwined with power. To write dialect, to transcribe an oral tale, or to mimic the cadences of a marginalized community’s speech is to assert the legitimacy of that community’s experience. It challenges the reader (often assumed to be from a dominant culture or at least fluent in its language) to adapt, to listen, and perhaps to confront their own preconceptions. The ethical thrust is toward recognition and respect: recognition of voices that colonial modernity tried to silence, and respect for forms of knowledge and art that come through ears and voices rather than the printed word alone. These postcolonial perspectives will inform our case studies, especially in Chapter 4’s sections on transnational and African American literature, where we will see these general principles enacted in specific texts.

\section{Modernism, Postmodernism, and the Changing Soundscape of Fiction}
While the previous sections have touched on historical context, it is useful to explicitly consider how approaches to sound differ between \textbf{modernist} and \textbf{postmodernist} writing, as this will frame the analysis of texts from different eras. The modernist period (roughly 1900–1945) was, as noted, a time of rapid technological change and urbanization, which dramatically altered everyday soundscapes. Authors of this era often felt that traditional literary techniques were inadequate to capture the new reality of honking automobiles, industrial machinery, gramophones playing recordings, radios broadcasting news and entertainment, and the general increase in the pace and noise of life. Modernist fiction is therefore marked by formal experimentation aimed at representing subjective experience in this changed world. As Angela Frattarola observes, modernist novelists used sound as a way to “bridge the distance between characters and to connect with the reader on a more intimate level” [oai_citation_attribution:31‡floridapress.blog](https://floridapress.blog/2018/12/04/modernist-soundscapes/#:~:text=In%C2%A0Modernist%20Soundscapes%3A%20Auditory%20Technology%20and,on%20a%20more%20intimate%20level) [oai_citation_attribution:32‡floridapress.blog](https://floridapress.blog/2018/12/04/modernist-soundscapes/#:~:text=These%20writers%20challenged%20ocularcentrism%2C%20the,the%20course%20of%20contemporary%20literature). By foregrounding auditory experiences (for example, a character’s acute awareness of a ticking clock or a distant train whistle), modernists tried to collapse the space between the reader’s world and the character’s inner world. Listening became a metaphor for empathy and immediacy: unlike the eye, which observes from a distance, the ear “receives” and is penetrated by sound, a process Frattarola describes as being “immediate and unifying” [oai_citation_attribution:33‡floridapress.blog](https://floridapress.blog/2018/12/04/modernist-soundscapes/#:~:text=These%20writers%20challenged%20ocularcentrism%2C%20the,the%20course%20of%20contemporary%20literature). 

In practical terms, modernist writers developed techniques like \textbf{stream of consciousness} and \textbf{free indirect discourse} which often have a sonic character. Stream of consciousness, used famously by Joyce and Woolf, attempts to follow the flow of a character’s thoughts, which can include sensory impressions and half-formed inner speech. The effect for the reader is akin to overhearing the internal monologue of the character, unmediated by an external narrator. This technique was in part a response to what Frattarola calls the “age of noise” and the influence of devices like the phonograph [oai_citation_attribution:34‡floridapress.blog](https://floridapress.blog/2018/12/04/modernist-soundscapes/#:~:text=She%20argues%20that%20the%20common,reproduction%20of%20the%20tape%20recorder). Just as someone might experience a phonographic recording as a voice speaking directly into their head, stream of consciousness gives the illusion of direct access to a voice. It is not coincidental that early psychological theories (like those of William James, who coined “stream of consciousness”) and technologies (recording, telephone) emerged around the same time—modernist literature synthesized these influences into a narrative form. The ethical implication here is that by adopting the perspective of a listening ear to a character’s mind, the modernist novel asks the reader to empathize deeply and non-judgmentally. Woolf, for instance, breaks down the barriers between characters’ consciousness in \textit{Mrs Dalloway}, often moving from one character’s thoughts to another’s fluidly, with the tolling of Big Ben or the roar of an airplane engine serving as an auditory bridge that momentarily unites them in shared time [oai_citation_attribution:35‡floridapress.blog](https://floridapress.blog/2018/12/04/modernist-soundscapes/#:~:text=These%20writers%20challenged%20ocularcentrism%2C%20the,the%20course%20of%20contemporary%20literature). This can be seen as an attempt to cultivate a broader human understanding: everyone in London hears the same clock, even if they interpret it differently, and the novel invites us to listen in on all those interpretations.

Modernism also saw a tension between \textbf{silence and sound}. In a world of growing noise, silence became precious and sometimes ominous. Writers like Beckett (who straddle modernism and the later absurdist movement) were as concerned with the failure of language and the impossibility of communication as with its presence. Beckett’s work often thematizes silence and pauses—on stage this is literal silence; on the page it’s the sense of nothing being said or the breakdown of meaningful speech. Modernist literature’s fascination with stream of consciousness was matched by an acknowledgement of the inexpressible; what cannot be put into words (or what sounds cannot be articulated) becomes a significant gap that the reader must also “hear.” Thus, modernist auditory aesthetics include the unheard and unspeakable—like the traumatic experience Septimus faces in \textit{Mrs Dalloway} (shell-shocked by war, he hears hallucinated voices and desperate silences that others cannot hear), or the profound final silence after the crescendo of Molly Bloom’s monologue in \textit{Ulysses}.

Moving to \textbf{postmodernism} (approximately 1960s onward), the situation shifts in some ways while continuing certain modernist trajectories. Postmodern literature inherited the modernist penchant for experimentation but often with a different attitude: where modernists were often earnest about capturing reality more authentically through experiment, postmodernists are more likely to playfully acknowledge the artificiality of any representation. In terms of sound, this means postmodern works might not strive to authentically convey a soundscape, but rather to use sound and noise metaphorically or metafictionally. A novel like Don DeLillo’s \textit{White Noise} (1985) is a prime example: it explicitly thematizes the barrage of media sounds, advertising jingles, and trivial communications that constitute late twentieth-century American life. DeLillo’s approach is often ironic; the “white noise” of the title refers to both the ever-present background radiation of technology and consumer culture, and to the existential fear of death that hums in the protagonist’s mind. In contrast to the modernist use of sound to connect and unify, postmodern sound is often about disconnection and overload. The ethics of listening in such a context becomes complex: when everything is noise, how do we discern a meaningful voice? The protagonist of \textit{White Noise}, Jack Gladney, struggles to truly listen to his family members amidst the constant interference of televised phrases and supermarket announcements that intrude into the narrative as disembodied voices. The novel suggests that modern individuals must filter and manage an excess of sonic information—an ethical challenge of a different sort, requiring mindfulness to not tune out what matters.

Postmodern narratives also frequently incorporate \textbf{multiple media and pastiche}, which can include sonic elements. For instance, a postmodern novel might include the lyrics of songs, transcripts of radio broadcasts, or the script of a play/film, blending these forms into the prose. This collage approach creates a text that is, in a sense, multi-voiced, but not in the Bakhtinian polyphony sense of deeply realized character voices. Instead, it’s the polyphony of culture’s many clashing sounds. Thomas Pynchon’s works like \textit{Gravity’s Rainbow} (1973) exemplify this with their frequent insertion of silly songs, slogans, and messages from various channels. The effect can be cacophonous and is often intended to satirize or critique the fragmentation of experience. The reader of postmodern fiction might find themselves in the position of a radio tuner, picking up bits of channels and trying to piece together coherence. Ethically, this can reflect the challenge of finding authentic meaning or moral direction in a world saturated with media and contested truths.

One area where modernism and postmodernism interestingly converge is in their engagement with \textbf{recording and broadcast technology}, though their tones differ. Modernist literature was fascinated by recording as metaphor (as we saw with phonographs inspiring stream-of-consciousness narrators “recording” their minds [oai_citation_attribution:36‡floridapress.blog](https://floridapress.blog/2018/12/04/modernist-soundscapes/#:~:text=She%20argues%20that%20the%20common,reproduction%20of%20the%20tape%20recorder), or Beckett’s later play \textit{Krapp’s Last Tape}, where an old man listens to reels of his younger voice). Postmodern literature frequently addresses television, radio, and the internet. The difference lies in reception: for modernists, technology was novel and often treated with a mix of hope and anxiety (e.g., could the gramophone make art immortal? Could the radio create a new global community of listeners?). For postmodernists, these technologies are taken for granted and often viewed with skepticism or dark humor (the TV in \textit{White Noise} spouts trivia even during intimate family moments, eroding authentic human communication). From a political standpoint, this also reflects a shift: modernist soundscapes were about the individual’s experience in a changing world, whereas postmodern soundscapes are about the collective drowning in simulacra and media noise. 

However, it’s important not to draw too rigid a line. Many contemporary writers build on both legacies. Toni Morrison, writing mostly in the postmodern period, employs modernist-like interior monologues and a keen ear for oral tradition, but also self-consciously addresses the gaps and silences of history (as in \textit{Beloved}, where the story of slavery is fragmented like a ghost story, acknowledging how trauma resists tidy narration). In other words, she merges a modernist depth of voice with a postmodern awareness of historical erasure and partial knowledge.

In conclusion, modernism and postmodernism offer two broad paradigms in which to situate literary sound. Modernism often treats sound as a path to deeper truth or unity of experience (despite recognizing noise and silence as challenges), whereas postmodernism treats sound/noise as objects of play, critique, and fragmented meaning-making. Both have rich ethical subtexts: modernist sound can be about empathy and breaking out of the isolation of consciousness, while postmodern sound can be about acknowledging multiplicity and resisting any single authoritative “signal” in the noise. The chapters that follow will draw on these literary-historical contexts. For example, when analyzing Joyce or Woolf, we’ll see the optimistic side of auditory experimentation—how capturing the music of thought was an attempt at honesty and connection. When analyzing DeLillo or other contemporary authors, we’ll examine the more skeptical side—how representing a world of endless chatter and buzz can question whether genuine communication is still possible or highlight which voices get lost in the din.

Having surveyed these theoretical landscapes—sound studies, narrative ethics, postcolonial theory, and the modernist/postmodernist contrast—we have a solid groundwork for the analytical chapters. The literature review has identified key ideas: the role of the listener in narrative, the significance of giving voice to the voiceless, the interplay of sound and power, and the historical evolution of literary soundscapes. The following methodology chapter will outline how these ideas will be operationalized in the thesis, and then we will proceed to hear the texts themselves in detail.

\chapter{Methodology}
This chapter details the methodological approach of the thesis, which combines traditional literary analysis with digital text analysis. The research questions require both interpretive depth—understanding nuanced literary techniques and themes—and breadth—recognizing patterns across multiple authors and time periods. As such, a mixed-method approach is employed. The primary method is qualitative close reading informed by the theoretical frameworks discussed in the literature review. A secondary, supporting method involves \textbf{corpus linguistics and computational analysis} applied to a curated selection of texts. This dual approach ensures that the argument remains grounded in the rich, context-sensitive reading of literature, while also benefiting from quantitative evidence where appropriate.

\section{Close Reading and Comparative Analysis}
Close reading is the cornerstone of literary scholarship and is particularly crucial for examining how sound operates in prose. Many of the phenomena under study—onomatopoeia, rhythm, dialogue, dialect, narrative voice—are subtle and context-dependent, often revealing themselves only through attentive analysis of passages. Each case study (Chapter 4) in this thesis was approached by first immersing in the primary texts to identify prominent auditory motifs and techniques. This involved reading novels and stories with a focus on any reference to sound or auditory perception: for instance, noting when a narrative describes a sound (like a clock chiming, a character’s tone of voice, background noise), or when the prose itself takes on sonic qualities (such as alliteration or repeated patterns that mimic a sound).

Annotations were made on the primary texts (either in print or using digital e-book tools) to mark instances of auditory imagery and significant dialogue. For example, in analyzing \textit{The Sound and the Fury}, sections of Benjy’s and Quentin’s monologues were marked for how they convey sound (Benjy’s heightened sensory impressions of auditory cues, Quentin’s obsession with the sound of his watch). In Toni Morrison’s \textit{Beloved}, instances of storytelling, song, and addressed dialogue (“Tell me, how did she get away?” and the call-and-response of the exorcism scene) were highlighted. This meticulous cataloging of evidence ensured that each interpretation in the analysis is backed by concrete textual detail.

A comparative lens was then applied. Because the thesis aims to link authors and eras, it was important to identify both unique uses of sound in individual works and shared patterns across works. To facilitate this, a simple matrix was used: a table listing each primary text (rows) against various sound-related categories (columns). Categories included: instances of onomatopoeia; musical references or structures; presence of oral storytelling within the narrative; dialect use; references to silence; metaphors of listening or deafness; and technology (mentions of radios, telephones, etc.). Filling out this matrix helped in detecting patterns. For instance, nearly every modernist text in the study had explicit onomatopoeic sequences (Joyce’s “Sirens” chapter famously contains strings of musical syllables, Woolf’s \textit{Between the Acts} includes a gramophone’s “scraping” sound rendered in letters). Meanwhile, many postcolonial texts showed heavy use of proverbs and oral narrative frames. By seeing these side by side, the methodology allowed for observations such as: modernist and postcolonial texts both value orality, but modernists often simulate interior monologue (the mind’s “voice”) whereas postcolonial writers often simulate communal oral storytelling (the village or family’s voice).

Close reading was also guided by the theoretical frameworks. For instance, when reading through Ralph Ellison’s \textit{Invisible Man}, Levinas’s idea of ethical listening (the obligation to hear the Other) was kept in mind, leading to an interpretation of the famous scene where the narrator gives an impromptu speech to a crowd: the novel’s crescendo of his voice and the crowd’s response can be read as a moment of being truly heard, swiftly followed by that voice being co-opted by an organization (the Brotherhood) that eventually muffles it. This theoretical lens sharpened the focus on why certain sound moments were pivotal: it wasn’t just that they were descriptively powerful, but they often carried ethical turning points (a character asserting themselves, or failing to be heard).

Throughout the analysis, a conscious effort was made to ensure that interdisciplinary connections did not overshadow the literary context. In practice, this meant that each literary work was first analyzed on its own terms (What is the role of sound in this story? How does it affect characters and reader? What artistic purpose does it serve?), and only then connected to broader themes or other works. This prevents a one-size-fits-all reading and respects each author’s unique style and historical situation. For example, the way James Joyce uses sound in 1922 Dublin is different from how Toni Morrison uses it in a 19th-century-set novel published in 1987, even if we might later talk about both in terms of giving voice to the voiceless. The methodology, therefore, emphasizes context: paying attention to when and where each text was produced, and what audience and purpose the use of sound might have had in that context.

\section{Corpus Linguistics and Computational Text Analysis}
To complement the close readings, a small corpus of texts was assembled for computational analysis. This corpus included the primary works analyzed and some auxiliary texts for baseline comparison (for instance, a few Victorian novels like Charlotte Brontë’s \textit{Jane Eyre} and Charles Dickens’s \textit{Bleak House} were included to contrast 19th-century practices with 20th-century modernism regarding sound description). The digital full-texts of these works were obtained from reliable sources (public domain texts from Project Gutenberg for older works, and e-book or scanned texts for later works when available). Using corpus analysis software (such as AntConc or Voyant Tools), several types of queries were run:

- **Keyword in Context (KWIC)**: Searches for specific sound-related terms (e.g., “sound,” “voice,” “listen,” “hear,” “silence,” “music,” “noise”) to see how frequently and in what contexts they appear in each text. The KWIC function lists each occurrence of a word with a snippet of surrounding text, which helped to quickly survey how each author employs those terms. For example, KWIC revealed that Virginia Woolf’s \textit{The Waves} uses the word “voice” dozens of times, often collocated with natural imagery (voices of birds, the sea, etc.), whereas in \textit{Invisible Man}, “voice” often appears in political contexts (the narrator worrying about finding his “true” voice or being the spokesperson for a cause).

- **Frequency counts and comparative ratios**: The total occurrences of key auditory words were normalized by text length to compare across works. This was not to quantitatively prove any point definitively (since word counts alone can be misleading), but rather to gauge emphasis. It confirmed, for instance, that Joyce’s and Woolf’s texts have a higher density of sound words than a sample Victorian novel, supporting the idea of a sonic intensification in literary modernism. It also showed that Toni Morrison’s \textit{Song of Solomon} has an unusually high frequency of words related to music (song, sing, guitar, etc.) compared to other contemporary novels, reinforcing our qualitative observations about its musical themes.

- **Concordance lines for dialect and onomatopoeia**: We attempted to use pattern searches to find non-standard spellings or strings of repeated letters that might indicate onomatopoeia or dialect. For instance, searching for double letters or exclamation markers can sometimes find comic-book-like sound words (“boom,” “zzz,” etc.) or elongated words (“soooo”). While this was a bit experimental, it helped spot a few things: in Hurston’s \textit{Their Eyes Were Watching God}, patterns like “‘Ah’” (for “I”) appear frequently, which is a dialect marker, and interjections like “Lawd” (Lord) or “Dat” (that) stand out, quantifying her use of phonetic dialect spellings. In Joyce’s works, strings like “tatta-tatty-tap” [oai_citation_attribution:37‡sparknotes.com](https://www.sparknotes.com/lit/ulysses/section11/#:~:text=The%20interspersed%20%E2%80%9Cjingle%E2%80%9D%20of%20Boylan%27s,of%20underlying%20rhythm%20section) were detected by searching for hyphens (since Joyce connects onomatopoeic syllables with hyphens or spaces). This provided examples to cite of Joyce’s sound-play.

- **Collocation and topic modeling**: To explore if sound-related words clustered with particular thematic words, collocation analysis was employed. For example, in \textit{Beloved}, the word “cry” often appears near words like “baby” or “dead” (reflecting the literal cries of the ghost baby and the metaphorical cry of the community). Such findings, while not surprising to a human reader, provided a systematic confirmation of themes: in this case, that the sound of crying (grief) is central to the novel’s depiction of haunting and memory. Topic modeling (an algorithmic grouping of words into topics) was tried on a set of novels to see if one of the emergent “topics” corresponded to sound. In a model with 10 topics trained on a mix of texts, one topic did emerge with high probability words like “voice, said, hear, silence, speak, words,” which indeed was prominent in texts like \textit{The Woman Warrior} and \textit{Invisible Man}. This computational result interestingly aligned with the intuitive notion that those books thematically revolve around speaking and being heard.

It is important to stress that these computational steps were \textit{not} the primary basis for interpretation but served as supporting evidence and sometimes as discovery tools. In several instances, the digital analysis highlighted something that merited a closer look in context. For example, the corpus analysis showed that in \textit{Invisible Man}, the word “music” appears frequently in the first chapter and the last (the battle royal scene and the epilogue with the jazz-inspired dream), prompting a closer re-reading of those sections to analyze the narrative function of music and noise at the novel’s beginning and end.

Another way corpus methods supported the study was by providing \textbf{counter-examples or controls}. If a certain pattern was thought to be significant, checking other texts helped determine if it was unique or common. For instance, the finding that Morrison uses many musical terms gains meaning when we see that another novelist of her time (say, Philip Roth or Saul Bellow in a quick check) did not use nearly as many. This suggests Morrison’s usage is a deliberate artistic choice rather than a general trait of late 20th-century fiction. In constructing an argument about “auditory ethics,” such context is valuable to avoid over-generalization.

Throughout the process, caution was exercised to not let the computational tail wag the analytical dog. If a quantitative result contradicted a qualitative sense, it was examined carefully. Sometimes the discrepancy was illuminating. For example, one might assume James Joyce uses a great many explicit sound words, but actually a tool showed that in \textit{Ulysses} the word “sound” itself isn’t extraordinarily frequent; rather, Joyce implies sounds through inventive language more than naming them. This clarified that not all engagement with sound will show up in raw word counts—reinforcing the necessity of close reading to catch the implicit or stylistic sonic elements.

\section{Interdisciplinary Synthesis and Ethical Research Practice}
After gathering both qualitative notes and quantitative data, the analysis phase involved synthesizing these findings within the theoretical frameworks. Each case study in Chapter 4 is structured to move from detailed observation to broader interpretation. For example, in discussing a modernist novel, the chapter starts with excerpts and narrative analysis of key passages (detailed work that might note the effect of a repeated onomatopoeic motif), then connects those findings to modernist principles or historical context (e.g., linking it to the influence of urban noise or a sound technology), and finally relates it to the concept of auditory ethics (what ethical or political point is being communicated through this use of sound—perhaps a challenge to readers to empathize with a consciousness or a critique of those who fail to listen).

The methodology also attends to the \textbf{ethics of research}. Since this thesis deals partly with minority and colonized voices, it was imperative to approach those texts with cultural sensitivity and awareness of one’s own position as a researcher. Whenever possible, writings by authors or scholars from the culture in question were consulted to avoid imposing an external interpretation that might miss nuances. For instance, in interpreting Hurston’s use of dialect, the thesis draws on her own essays (like “Characteristics of Negro Expression”) to understand her intent, rather than solely on later academic theory. Similarly, readings of postcolonial texts were checked against postcolonial critics from those regions.

In terms of citation and academic integrity, all uses of sources—whether theoretical texts or digital tools outputs—are documented. The thesis adheres to the Chicago citation style for all bibliography entries and uses the specified bracket notation for in-text referencing of sources gathered during research (e.g., referencing the line from a digital archive or scholarly source as 【source†lines】 for transparency). This ensures that claims such as “Ellison said his approach to writing is through sound” or “Woolf’s onomatopoeia aimed to render unmediated sound” are backed by verifiable references [oai_citation_attribution:38‡davidpublisher.com](http://www.davidpublisher.com/Public/uploads/Contribute/6684fb270b00c.pdf#:~:text=state%20with%20remarkable%20pride%20that,approach%20to%20writing%20is%20through) [oai_citation_attribution:39‡floridapress.blog](https://floridapress.blog/2018/12/04/modernist-soundscapes/#:~:text=a%20listener%E2%80%99s%20headspace%2C%20inspired%20modernists,reproduction%20of%20the%20tape%20recorder).

Finally, a note on limitations: the methodological choice to integrate computational analysis is meant to broaden perspective, but it cannot capture everything. Not all aspects of “sound” in literature are text-searchable (tone, for instance, is something one feels in the prose, not a keyword). Moreover, the corpus selected, while diverse, is not exhaustive of all literary traditions. There is an unavoidable focus on Anglophone literature; works in other languages are outside the scope, though they would provide valuable comparative insight. These limitations are acknowledged so that conclusions are understood as pertaining to the corpus at hand, with suggestive (but not automatic) extension to other contexts. Future research could expand the computational element to a larger multilingual corpus or incorporate reader-response studies (asking real readers about their auditory imagination during reading) for further triangulation.

In summary, the methodology chapter establishes that this thesis uses a blend of deep reading and wide-angle analysis. The interpretive work remains at the center, ensuring that the argument about auditory ethics is rooted in actual literary practice. The computational tools serve as both a magnifier for patterns and a safety net against purely subjective impressions. Together, they enable a robust examination of how sound functions across different authors, and how those functions carry ethical and political meaning. With this methodological framework in place, the thesis now turns to the case studies, where these methods are applied to specific literary works and groups of works.

\chapter{Case Studies}
\label{chap:case-studies}
The following case studies are organized into thematic and chronological groupings to highlight both the evolution of literary soundscapes and the distinct ways different authors deploy sound for ethical and political ends. Section 4.1 examines \textbf{modernist fiction}, where experiments with representing consciousness and perception often foreground auditory experience. Section 4.2 moves to \textbf{postcolonial and transnational narratives}, focusing on how writers from colonized or diasporic backgrounds reclaim orality and voice. Section 4.3 centers on \textbf{African American literature}, with its rich intertwining of music, speech, and storytelling as tools of cultural survival and resistance. Section 4.4 discusses \textbf{postmodern and contemporary fiction}, where the emphasis shifts to noise, media saturation, and the fragmentation of voice in late twentieth-century life. Each section provides close readings of representative works and considers how those works illustrate the concept of “auditory ethics.” Throughout, connections will be drawn back to the theoretical concepts introduced earlier, with attention to the specific historical and cultural contexts of each case.

\section{Modernist Soundscapes: Joyce, Woolf, and their Contemporaries}
The modernist era produced some of the most famously “auditory” prose in English literature. Writers such as James Joyce and Virginia Woolf radically broke from linear storytelling, seeking to represent the inner workings of the mind and the flux of lived experience. In doing so, they often treated sound not just as one aspect of setting or description, but as a structural principle and thematic focus. This section explores how Joyce and Woolf, along with figures like Dorothy Richardson, William Faulkner, and Samuel Beckett (who carried modernist tendencies into mid-century), crafted literary soundscapes that engage ethical questions about perception, empathy, and the limits of language.

**James Joyce’s Auditory Experiments:** Joyce’s \textit{Ulysses} (1922) is frequently cited as a pinnacle of modernist experimentation, and its “Sirens” episode (Chapter 11) is a virtuoso performance in making written words sing. Set in a Dublin bar with two flirting barmaids, a piano tuner, and various patrons, the episode is structured akin to a musical fugue. Before any coherent narrative unfolds, Joyce gives the reader an overture of sorts: a series of seemingly disjointed phrases and onomatopoeic sounds (“Bronze by gold heard the hoofirons, steely-ringing” and “Ta ta ta… Clang.  Tabid.  Kran, kran, kran” are examples) [oai_citation_attribution:40‡jstor.org](https://www.jstor.org/stable/25486536#:~:text=The%20Sirens%20Episode%20as%20Music%3A,10%20complemented%20by) [oai_citation_attribution:41‡sparknotes.com](https://www.sparknotes.com/lit/ulysses/section11/#:~:text=The%20interspersed%20%E2%80%9Cjingle%E2%80%9D%20of%20Boylan%27s,of%20underlying%20rhythm%20section). These are the “musical motifs” that will recur throughout the chapter. By disrupting conventional syntax and meaning, Joyce forces the reader to abandon a purely visual decoding of text and instead to sound out the syllables, to \textit{listen} to the language. The effect is that one enters the scene through sound—laughter, tapping, singing—before situating who or what produces those sounds. Ethically, this disorientation and subsequent reorientation mirror one of Joyce’s themes: the difficulty of truly understanding others’ experiences. We are initially like the blind piano tuner who appears in the scene, hearing signals and feeling our way forward. As the episode progresses, the narrative coalesces around two songs being sung in the bar and the internal ruminations of Leopold Bloom (who is present, largely silent, internalizing his emotional hurt over his wife’s infidelity). The prose is laced with musical terminology and onomatopoeia: Bloom notes the “jingle” of Boylan’s approaching car (Boylan being his wife’s lover) and the “tap” of the blind tuner’s cane [oai_citation_attribution:42‡sparknotes.com](https://www.sparknotes.com/lit/ulysses/section11/#:~:text=Ulysses%20Episode%20Eleven%3A%20%E2%80%9CSirens%E2%80%9D%20Summary,of%20underlying%20rhythm%20section); these sounds acquire symbolic meaning—the jingle a reminder of betrayal, the tapping a reminder of human vulnerability. By making the reader acutely aware of these auditory details, Joyce invites us to inhabit Bloom’s sensitive, almost over-stimulated consciousness. 

The politics of sound in “Sirens” also play out in terms of gender and power. The barmaids’ flirtatious laughter is described in musical terms (“the first high notes…trilling”); they are likened to the Sirens of mythology whose songs lure men [oai_citation_attribution:43‡campuspress.yale.edu](https://campuspress.yale.edu/modernismlab/the-sirens/#:~:text=The%20first%20sort%20of%20music,associated%20with%20the%20Sirens%27). Yet, Joyce subverts the male gaze by focusing on the \textit{sound} of the women rather than their visual appearance. In doing so, he gives them a kind of agency through voice—they are not just objects to be looked at, but subjects who sing and influence the emotional atmosphere. Bloom, who often empathizes with women in the novel, listens to their singing with appreciation but also personal sadness, thinking of his wife’s musical past and present. The ethical listening here is twofold: the characters listening to each other (Bloom listening without resentment to the same music that Boylan enjoys, which is a very generous act in context), and the reader listening to all. Joyce’s narrative technique forces us to process multiple sounds and voices at once, akin to how one must in a polyphonic musical piece. In doing so, he perhaps cultivates in the reader what scholar Liz Lipscomb calls “sympathetic hearing” – an ability to hold conflicting emotional currents together. Bloom’s quiet heartbreak and the barmaids’ mirth occur simultaneously; the reader hears them both, fostering a compassion for Bloom that doesn’t vilify the women or the joyful music. Joyce’s auditory ethics lies in this insistence on polyphony without moral simplification.

**Virginia Woolf’s Choruses and Silence:** Virginia Woolf’s novels also deeply engage sound, though often in a less flamboyant way than Joyce’s. Woolf was interested in the concept of a collective consciousness and how individuals resonate with one another. In \textit{Mrs Dalloway} (1925), one of Woolf’s techniques is to use the striking of Big Ben as a rhythmic backbone for the novel. The clock’s chime is described in rich terms (“The leaden circles dissolved in the air”) and each character reacts to it in their own way. Woolf noted that hearing is a sense that can unite an otherwise disparate crowd; indeed, as Big Ben tolls noon, Clarissa Dalloway pauses in her party preparations while miles away Septimus Warren Smith, a shell-shocked war veteran, also hears the bell just before tragically taking his life. The shared sound reminds readers that these characters, though one is a wealthy society hostess and the other a traumatized outsider, occupy the same city and moment. Woolf’s narration flows from Clarissa’s point of view to Septimus’s around that moment, using the sound as a bridge. This can be read as an ethical statement about societal connection: even if Clarissa does not know Septimus, Woolf the author ensures that his story is heard by the reader and by Clarissa (indirectly, via another character) later on. Woolf “breaks with a literary tradition that regarded the sounds of modernity as a distraction” and instead embraces them as meaningful [oai_citation_attribution:44‡jstor.org](https://www.jstor.org/stable/10.13110/criticism.59.4.0565#:~:text=RE,as%20a%20distraction%3B%20rather%2C%20in). The onomatopoeic rendering of sound in Woolf (for example, the recurring “swish” of the curtains in \textit{Mrs Dalloway}, or the famous opening of \textit{The Waves} where the chorus intones “Oo” to mimic the sound of the sea) serves to pull the reader into an auditory imaginative space.

In \textit{The Waves} (1931), Woolf’s most experimental novel, six characters’ interior monologues are intercut, with each often describing the same moment (like a sunrise) in their distinct voice. This novel is like a chamber music piece, each voice an instrument that sometimes plays solo, sometimes in unison. Woolf explicitly wanted to capture “the singing of the mind” – an almost musical essence of consciousness. Reading \textit{The Waves}, one becomes aware of how pattern and rhythm in sentences create a mood: Bernard’s reflective looping phrases, Rhoda’s broken, disjointed utterances, and Susan’s earth-bound exclamations all have different cadences. Woolf achieves a kind of \textbf{aural characterization}. The ethics in \textit{The Waves} might be seen in how it invites the reader to value each voice, to see reality as a composite of many subjective experiences rather than one dominant narrative. By giving each character chapters of pure soliloquy, Woolf enacts a literary justice: everyone gets their say, their voice fully unfurled. There is no overt plot to prioritize one over the other; instead, the book’s meaning emerges from the harmony and dissonance of these voices.

Woolf also uses \textbf{silence and its rupture} as a narrative tool. In \textit{Between the Acts} (1941), her last novel, which centers around a village pageant (a play performed by villagers), there is a phonograph that malfunctions—at one point, instead of playing music between scenes, it emits an “agonizing silent moment” followed by a loud, unexpected sound that startles everyone. This device underscores the fragility of communication and performance. Just when the villagers are supposed to be joined in a communal auditory experience (listening to music together), technology fails and they are thrown into awkward silence, which in turn forces them to confront themselves and each other. The pageant itself in the novel is a metaphor for national history and unity, but Woolf peppers it with moments where the audience either doesn’t hear well (wind carrying off the actors’ voices) or hears too well (the clink of cutlery from an outdoor tea interfering). In these moments, the reader is made conscious of listening as an active, sometimes strained effort. Woolf’s modernist auditory ethic perhaps lies in illustrating that true understanding (of art, of each other) requires attentive listening, and that mishearing or noise can either alienate or, if recognized, become part of the shared human condition.

**Beyond Joyce and Woolf:** Other modernist and early 20th-century writers also contribute to this theme. Dorothy Richardson’s \textit{Pilgrimage} series (1915–1938), often credited as the first sustained use of stream of consciousness in English, pays great attention to the sounds of London as experienced by a woman living independently. Richardson’s protagonist Miriam notices, for instance, the distinct “accents” of different neighborhoods, the clatter of horse-drawn buses giving way to motorized ones over time, etc. These details serve as both setting and a commentary on change; the sound of modernization is relentless, and Miriam’s reflections on it parallel her own internal changes. The ethical dimension comes through Miriam’s at times overwhelmed reactions—Richardson, via sound, evokes sympathy for the modern individual (particularly a woman) bombarded by external stimuli and expectations.

William Faulkner’s \textit{The Sound and the Fury} (1929) deserves mention given its very title highlights sound. The phrase comes from Macbeth—“a tale told by an idiot, full of sound and fury, signifying nothing”—which Faulkner uses ironically. The novel’s first section is narrated by Benjy, a man with an intellectual disability who cannot speak. Instead, he registers the world through sensations, especially sound (he is acutely sensitive to the tones of people’s voices and the sounds of golfers shouting by the pasture). Faulkner’s challenge was to convey Benjy’s experience without giving him conventional language. He does so through a kind of impressionistic monologue that jumps in time via triggers. Notably, certain sounds catapult Benjy into memory: the sound of his sister Caddy’s name makes him moan mournfully, church bells might remind him of a childhood moment. Here, sound is tied to trauma and loss. The ethical provocation is strong: we as readers are asked to piece together Benjy’s story despite his incapacity to narrate logically. We must become empathetic listeners to a person who, in the fictional world, is considered almost sub-human by his family. Faulkner once said that in writing Benjy’s section, he tried to imagine the world of a mentally disabled man for whom time held no meaning and who understood events only by their emotional sound and feel. This is a profoundly humane exercise, compelling the reader to inhabit a radically different consciousness. By giving Benjy the first section of the novel (and thus the first “voice” we encounter), Faulkner inverts the usual hierarchy—normally, someone like Benjy would be voiceless, an object of pity or a side character. Instead, he speaks (albeit in a unique literary way) and the world must listen, however difficult that is. That difficulty is the reader’s ethical test and training: learning to listen to Benjy trains us to listen differently to the other narrators and to question the values of a society that marginalizes voices like his.

Samuel Beckett, though often associated with the post-war absurdist movement, was in many ways a late modernist, especially in his prose trilogy (\textit{Molloy}, \textit{Malone Dies}, \textit{The Unnamable}, 1951–53). Beckett explores the extreme limits of voice and story—his narrators are often disembodied voices talking into the void. In \textit{The Unnamable}, the narrator famously says, “I can’t go on, I’ll go on,” encapsulating the tension between silence and the compulsion to speak. Beckett’s novels reduce the environment to almost nothing, so sound (the voice’s own sound, and occasional remembered sounds) becomes the last tether to existence. In these works, we might say the politics of sound becomes internal: the character’s struggle to have a voice at all, to not fall into silence (which in Beckett often symbolizes oblivion or death). The ethical gesture from Beckett to the reader is to ask for patience and compassion for narrators who themselves question the value of their words constantly. It’s as if Beckett anticipates an age (our age) of extreme isolation and asks: can we still listen to the barest, most frail human voice, even when it is rambling, incoherent, or despairing? If we can, perhaps there is hope in that act of listening itself.

In summary, modernist writers revolutionized literary form in part by making prose a vehicle for auditory experience, thereby heightening immediacy and emotional resonance. Joyce turned text into symphony; Woolf turned it into a choral meditation; Faulkner used it to articulate the voiceless; Beckett deconstructed it to the pure act of speaking/hearing as existence. Each of these innovations carries ethical implications. They challenge readers to become listeners, to attend to multiple voices without prejudice, and to appreciate the beauty and pain carried in sound. They also often critique the social order indirectly: by focusing on sound, they bypass the visual markers of identity (like appearance, race, class signifiers) and find a more universal or more sincere register of human experience. In doing so, they open a democratic space in the narrative—one where even an idiot’s tale “signifies” something, after all, and where the submerged voices (women, the poor, the traumatized) can surface in the auditory mix. This modernist foundation sets the stage for later authors, who would take the idea of giving voice further, explicitly into the realms of colonized and marginalized communities, which we turn to next.

\section{Postcolonial Voices and Orature: Achebe, Rushdie, Kingston, and Others}
In the wake of modernism and mid-century upheavals, literature from formerly colonized nations and diasporic communities gained global prominence. Postcolonial and transnational authors often write in the language of the former colonizer (English, French, etc.) but infuse it with the rhythms, sounds, and narrative structures of their native or ancestral cultures. A central concern of postcolonial literature is the reclamation of \textbf{voice} – asserting the right to speak and be heard after periods of enforced silence or misrepresentation under colonial rule. In this context, integrating oral traditions (“orature”) into written form becomes a powerful literary strategy. This section looks at how authors like Chinua Achebe, Ngũgĩ wa Thiong’o, Salman Rushdie, and Maxine Hong Kingston use sound and voice to politicize their prose, bridging the gap between oral and written, and inviting readers into an intercultural act of listening.

**Chinua Achebe and the Voice of the Village:** Achebe’s \textit{Things Fall Apart} is a foundational text of African literature in English. One of Achebe’s achievements is to make the Igbo world intelligible to international readers without diluting its cultural distinctness. Sound plays a key role in this. The novel opens not with English epigraphs or explanations, but with an Igbo conversation and the aforementioned proverb about proverbs being the palm-oil of conversation [oai_citation_attribution:45‡impactnetwork.org](https://www.impactnetwork.org/latest-news/proverbs-are-the-palm-oil-with-which-words-are-eaten#:~:text=I%20was%20looking%20back%20at,oil%20with%20which%20words%20are). Right away, this signals that the mode of communication valued in Umuofia (the fictional village) is oral performance – storytelling, proverbs, oratory. Throughout the book, important moments are marked by \textbf{public speaking and listening events}: for example, the community meeting in the marketplace is depicted with attention to how speakers modulate their voices and how the crowd responds in grunts or applause. When Okonkwo, the protagonist, violates social norms by insulting a man during a meeting, the narrative emphasizes the collective gasp and the stunned silence that follow – the sound of public censure. By showing these dynamics, Achebe places the reader in a listening position akin to a member of the community, gradually teaching us the “sound culture” of the Igbo, where certain drums signal war, certain cries signal mourning, and silence can signal dissent or shame.

One of the most striking uses of sound is during the \textbf{funeral of Ezeudu}, a village elder. The night is alive with the steady beating of drums and the intermittent blasts of a metal gong, along with a “voice” that seems to come from beyond – a egwugwu (ancestral spirit) speaking through a masquerader. The scene is both immersive and foreboding (it’s where Okonkwo’s gun accidentally explodes and kills a boy, a turning point). Achebe writes the sound in a rhythmic, repetitive prose, almost mimicking the drums: “Gome, gome, gome, gome went the gong, and a powerful flute blew a high-pitched blast” (paraphrasing). By capturing the layered soundscape, Achebe doesn’t just add realism; he positions the reader to feel the emotional weight that sound carries. The drumming at a funeral is a communal heartbeat of grief and respect, something beyond words. When that sound is abruptly replaced by screams after the accident, the emotional register flips from solemnity to horror. We “hear” the village’s agony, not just read about it. This is auditory ethics in that we are made to \textit{witness}, through sound, the impact of Okonkwo’s action on the community.

Achebe’s narrative method also uses an omniscient voice that at times takes on the quality of an \textbf{oral storyteller}. There are moments when the narrative voice addresses the reader almost like a fireside raconteur, saying things like “As the elders said…” followed by a proverb. This draws the reader into a role akin to a listener in an oral audience, acknowledging that what is being passed on is cultural wisdom, not just plot. It closes the distance between the literate, presumably Western reader and the oral, Igbo source of the tale, fostering a sense of cross-cultural communication. The politics here is making the English language carry Igbo thought and sound patterns, thereby “decolonizing” the language from within. It challenges any notion that to write in English one must write like an Englishman; instead, Achebe makes English sing to an African tune.

**Ngũgĩ wa Thiong’o and Orature on the Page:** Ngũgĩ’s novel \textit{Devil on the Cross} (1982), originally written in Gikuyu, is explicitly framed as an oral narrative. The book begins with a frame narrator saying that the story was told to him by a man (the Gicaandi Player) on the run, who had in turn performed it. Each chapter begins with a direct address as if the narrator is speaking to an audience, using phrases like, “Now listen, I will tell you what happened to Jacinta…” and often interjects with commentary or questions to the audience (“What would you have done in her place, I ask you?”). The effect is that the novel reads like a transcript of a performance. It includes songs and chants (with verse laid out on the page) that characters perform during the central “Devil’s Feast” event (a grotesque parody of a modern get-rich-quick contest). 

The translation into English retains many Gikuyu words and phrases, often unitalicized and untranslated in text, expecting the reader to either understand from context or learn. This insistence on bilingual texture is part of the auditory politics: it suggests that the fully “heard” version of the story would mix languages as the storyteller might sing a Gikuyu song and then explain it in Gikuyu, and perhaps someone might translate bits for an outsider. By not smoothing that over, Ngũgĩ forces the English reader into the position of needing a kind of interpreter or at least humble acceptance that not everything will be catered to their ear. It’s a reversal of colonial linguistic power.

Ngũgĩ’s heavy use of onomatopoeia and ideophones (vivid words that evoke sound or movement, common in African languages) also challenges English norms. For example, at one point he describes a character’s fancy car with a phrase like “zooming zinjiririi!” to mimic the revving engine, or the crowd’s murmurs with a string of syllables. These may seem unusual in an English novel, but they convey a sense of immediacy and humor that aligns with oral storytelling, where a performer might use sound effects vocally to entertain the audience. Here on the page, they serve to break the illusion of silent print, making us imagine the noise.

From an ethical perspective, Ngũgĩ uses these techniques to critique Kenyan society. By adopting the voice of a traditional storyteller, he casts a critical eye on modern corruption and greed, implying that the wisdom of the past (conveyed in stories and songs) is needed to judge the present. For instance, he includes a song about the evils of money in the midst of the Devil’s Feast, functioning as a Greek chorus guiding the reader’s moral interpretation. Listening to the story thus becomes not a passive act but a call to consciousness. This echoes the African oral tradition where stories often have didactic functions and the audience is expected to draw lessons. Ngũgĩ expects his reader to hear the anger and urgency in his narrative voice, which explicitly calls for resistance against neocolonial exploitation.

**Salman Rushdie and the Polyglot Voice:** Rushdie’s \textit{Midnight’s Children} (1981) is an example of how postcolonial narratives can embrace a chaos of sound to reflect the chaos of history. The narrator, Saleem Sinai, frequently addresses his story to an implied listener, Padma. He is conscious of his role as storyteller: “I must work fast, faster than Scheherazade, if I am to end up meaning – yes, meaning – something,” he says at one point (paraphrase). This self-conscious orality is part of the charm and technique. Saleem’s narrative is peppered with the sounds of India: the cries of vendors, fragments of popular songs, the fanfare of political rallies. Rushdie does not use conventional indicators for non-English words either, letting the cadence of Indian English flow. For example, an Anglo-Indian character might say in dialogue, “What to do, yaar? It is like that only,” and Rushdie will let that stand without footnote. The effect is that readers become attuned to a certain musicality and meaning even if the grammar seems odd – essentially, readers learn to hear Indian English by reading Rushdie, just as one picks up an accent’s patterns by ear. 

A key motif in \textit{Midnight’s Children} is Saleem’s telepathic connection with all other children born at midnight on India’s independence day. He imagines this as a “conference” of voices in his head, which he attempts to organize. Initially, it’s a cacophony – children from all parts of India, speaking different languages and with different concerns, all talking at once. Saleem as the narrator tries to impose some order (much like India’s government attempted to forge unity out of diversity). He even describes the experience in auditory terms: some voices are like whispers, some like shouts; some speak Urdu, some Tamil, some in tribal dialects. In writing these telepathic scenes, Rushdie essentially creates a literary sound collage, albeit with only hints at actual speech (since he can’t literally write in all those languages at once for an Anglophone book). He achieves it by peppering Saleem’s recounting with phrases in different languages and descriptions of their tones. Importantly, one of the loudest voices in that chorus is that of Shiva, Saleem’s nemesis, who is abrasive and speaks in a coarse slang, representing the repressed bitterness of those left behind. Thus, the Midnight Children’s Conference can be read as an allegory of free speech and democracy – all voices are present, but can they listen to each other? It fails eventually, mirroring how the ideals of the new nation falter. 

Rushdie’s novel suggests that telling the story of a nation (or personal story entwined with national history) requires polyphony; no single voice or style can capture it. Saleem’s narrative jumps in time and frequently rewinds or revises events (Padma often scolds him for inconsistencies, like a live audience might). This invokes the oral tradition of iterative storytelling – stories evolving with each telling, listeners interjecting or requesting clarification. In a written text, these would normally be edited out, but Rushdie leaves them in to simulate the storytelling session. The result for the reader is an active engagement: one is not just reading for what happens, but listening to how Saleem decides to reveal it, always aware that he could be embellishing or forgetting. The ethical layer here is about the nature of historical truth and memory. Rushdie implies that a single official history is suspect (just as a single authoritative narrative is a kind of authoritarianism), and that a truer, richer understanding comes from acknowledging a multitude of perspectives – even if it’s confusing and contradictory. By inundating us with sounds and voices, Rushdie forces a kind of critical listening; we have to discern, like sifting through a noisy bazaar, what is important and what it signifies.

**Maxine Hong Kingston and the Voice of Silence:** Kingston’s \textit{The Woman Warrior} (1976) is a blend of memoir and myth, subtitled “Memoirs of a Girlhood Among Ghosts.” It’s deeply concerned with the clash and blending of Chinese oral culture and American life. Kingston structures the book in five parts, each centered on a female figure whose story was often silenced. The very first chapter, “No Name Woman,” recounts the tragic tale of Kingston’s aunt, who became pregnant out of wedlock and ultimately killed herself and her baby. Kingston’s mother told her this story once in a hushed, strictly secret tone (“You must not tell anyone,” the memoir famously begins) [oai_citation_attribution:46‡impactnetwork.org](https://www.impactnetwork.org/latest-news/proverbs-are-the-palm-oil-with-which-words-are-eaten#:~:text=I%20was%20looking%20back%20at,oil%20with%20which%20words%20are). Immediately, we are introduced to an ethic of silence – this is a story that is not to be repeated, effectively erasing the aunt’s voice from history. Kingston, by writing it, breaks that silence. She even dramatizes the difficulty of doing so: since she doesn’t know the whole truth, she narratively “talks story” (a Chinese expression for speculative, elaborative storytelling) to imagine her aunt’s thoughts and feelings, effectively giving the aunt a voice in the text that she was denied in life. This is a striking example of using narrative to right an ethical wrong: filling silence with voice as an act of justice or remembrance.

Kingston’s prose frequently incorporates reported speech from her mother’s talk-story sessions, indicated by phrases like “my mother said” but often it will shift into a kind of first-person folktale mode (e.g., telling the legend of Fa Mu Lan, Kingston writes parts of it in first person as if she is the warrior woman). These shifts mimic the oral tradition of becoming the character when telling a story, often using dramatic present tense and voicing their dialogue. On the page, Kingston sometimes doesn’t use quotation marks for these stories, blending them into her own narrative. This can be momentarily disorienting – whose voice are we hearing? This confusion is deliberate: it mirrors Kingston’s own childhood confusion between the myths her mother narrated and the reality she lived in California. The boundaries between narrator and character voice blur, just as Kingston felt the boundaries between herself and the Fa Mu Lan legend blur (famously, she daydreams that she is Fa Mu Lan). 

From a sound perspective, Kingston’s work is unique in translating a tonal, rhythmic Chinese storytelling style into English. She uses repetition and incantatory phrasing reminiscent of oral epic. For instance, in the Fa Mu Lan section, certain images and lines repeat (“Eyes wide open, eyes wide open” when describing a peasant’s corpse, echoing through the text). Such repetition works as it does in oral literature – to hammer home imagery, to create a refrain that the “listener” anticipates, and to generate a poetic rhythm. Additionally, Kingston’s dialogue captures the cadence of Chinese-English bilingual speech. Her mother’s voice in the memoir often switches from Chinese (transliterated or described) to broken English. Kingston sometimes literally writes out Chinese words (using English phonetic spelling) and then gives their translation or leaves them for context. One memorable scene is when her mother calls her children “stupid” and “garbage” in both languages, out of frustration – Kingston processes the harsh sounds of the Chinese scolding and the bluntness of the English equivalents, demonstrating how they hit her ears. 

The most auditory chapter is “Shaman,” which describes her mother’s experiences in medical school in China and is full of sounds: chants to exorcise ghosts, the chatter of dormitory girls, the quiet of studying by oil lamps. Her mother, Brave Orchid, later in America hears “ghosts” in every appliance (the telephone’s ringing is a “ghost’s wail” to her, she calls a persistent milk delivery man a “milk ghost” because he keeps knocking). This framing, calling foreigners “ghosts,” flips the script: to her immigrant mother, it’s the Americans who seem unreal and whose voices are eerie (in effect dehumanizing them the way immigrants often feel dehumanized – a subtle political critique). Kingston translates some of Brave Orchid’s ghost exorcism chants and songs for us, giving a glimpse of a sound-based ritual that is alien to American readers, but she contextualizes it with such empathy and matter-of-factness that we accept it as part of Brave Orchid’s reality.

Ethically, Kingston’s project is about making a space for the voices of Chinese women (her mother, aunt, even herself) that mainstream (male, Western) history left out. She does it by blending their oral testimonies and stories with her own written voice. It’s an act of \textbf{re-presenting} oral culture to a new audience, requiring that audience to listen across cultural and linguistic boundaries. The constant code-switching and narrative shifts instruct the reader in bicultural literacy – effectively, how to hear a Chinese story within an English text. At the end of the book, Kingston shares an evocative metaphor: her mother tells her a final story about a poet who learns to make songs out of the clashes and noises of everyday life (saws, machines) – a story about how art can come from dissonance. Kingston ends by reflecting that she has outgrown some of her childhood fear of those stories and can now carry them forward. She metaphorically equates “voice” with survival and continuity. The book’s closing image is that of transforming discord into music, which encapsulates the auditory ethic: the immigrant experience (full of painful “clashes”) can be shaped into a song (a narrative) that connects past and present, self and community.

In summary, postcolonial and transnational prose often operates as a kind of recorded orality. Authors use sound—literal and figurative—to recover cultural identity and to challenge power. Whether it’s Achebe’s proverbs and communal voices, Ngũgĩ’s storyteller performance, Rushdie’s babble of a newborn nation, or Kingston’s braid of myth and memory—the act of writing these sounds is inherently political. It asserts that these cultures need not conform to Western literary norms; instead, they expand literature to be more inclusive of diverse ways of storytelling. For readers, the experience is ear-opening: we must learn new sound-symbol relationships, accept bilingual punning or un-translated words, and respect that understanding might require effort akin to learning to appreciate a new music. The reward is a richer, more polyphonic narrative universe where formerly marginalized voices speak (or sing, or shout) on their own terms.

\section{Sonic Narrative and Racial Identity: African American Literature and Music}
African American literature provides some of the clearest examples of how sound—particularly voices, dialect, and music—carries ethical and political significance. Because of the history of slavery and racial oppression in the United States, African American writers have frequently engaged with the legacy of oral culture (spirituals, folktales, preaching) and the nuances of black speech as a means of resistance and affirmation. Moreover, the idioms of jazz, blues, and other black musical forms have influenced literary style, not just as subject matter but as structural and linguistic inspiration. This section focuses on how writers like Zora Neale Hurston, Ralph Ellison, and Toni Morrison use sound and music in their narratives to explore identity, community, and power.

**Zora Neale Hurston: “Their Eyes Were Watching God” and the Sound of Folk Speech:** Hurston was both a novelist and a trained anthropologist/folklorist. In her 1937 novel \textit{Their Eyes Were Watching God}, she made a then-bold choice: to write most of the dialogue in authentic black Southern dialect. For example, the protagonist Janie and her friend Pheoby converse in lines like: “Ah done been tuh de horizon and back and now Ah kin set heah in mah house and live by comparisons” (Janie’s words towards the novel’s end, stating her experience) [oai_citation_attribution:47‡rupkatha.com](https://rupkatha.com/toni-morrisons-a-mercy/#:~:text=an%20Understanding%20of%20the%20Narratives,buried%20self%20suppressed%20into%20oblivion). Hurston spells words phonetically (“tuh” for “to,” “kin” for “can,” etc.) and uses grammatical constructions typical of spoken black English (“Ah done been” etc.). At the time, some critics (including black writers like Richard Wright) criticized Hurston for “minstrelizing” her characters or pandering to white audiences by showing dialect. However, Hurston’s intent was to celebrate the poetry and validity of her community’s speech. She wrote in an essay that the dialect of African Americans is rich with metaphor and nuance, and she wanted to capture its music on the page.

Reading Hurston’s dialogue, one cannot avoid “hearing” it—she effectively thwarts silent, abstract reading. The reader must sound out or at least subvocalize the words to understand them, which has the effect of bringing the characters’ voices alive in one’s imagination. This was part of her narrative ethic: to assert that this way of speaking is legitimate literature, that the souls and intellects of her characters shine through their language. Janie’s journey, framed as telling her life story to Pheoby (the novel both begins and ends with scenes of storytelling on the porch), reinforces the oral narrative tradition. Janie speaks her life into meaning. The very title “Their Eyes Were Watching God” comes from a line delivered by the narrator during a hurricane scene, but its resonance is felt through Janie’s perspective—when nature’s fury drowns out all human sound (as the hurricane does), people are left gazing at the sky in silence, “their eyes...watching God.” Here the absence of human sound in the face of the storm’s roar is telling: it’s a moment of awe and humility beyond language. But immediately after, the narrative resumes with intense sound imagery (Janie calls out to her husband Tea Cake against the roaring water, etc.). Hurston, by emphasizing dialect throughout, makes that brief loss of voice stand out sharply, underscoring the precariousness of her characters’ lives.

Hurston’s use of call-and-response patterns in group scenes also mimics African American oral culture. On the porch of Joe Starks’ store in Eatonville, men engage in playful verbal duels (“signifying” or “playing the dozens”), to the entertainment of the crowd. Hurston writes these scenes almost like a mini-script, capturing the pacing—each man’s retort one-upping the last, and the crowd’s laughter interjecting. One can almost hear the applause and foot-stomping. This performance aspect is key: Hurston shows that in this black community, mastery of language (even just witty insults) earns respect. It flips the social script: these characters, who in the broader white society might be voiceless or disregarded, have a space where their voices reign. By letting the reader witness these verbal contests, Hurston invites an appreciation of that cultural space. It’s an ethical move in that it insists on the humanity and creativity of her characters, countering stereotypes of black inferiority that often hinged on mocking their speech. Hurston essentially says: you call it broken English; I call it a new art form.

**Ralph Ellison: “Invisible Man” and the Jazz of Narration:** Ellison’s \textit{Invisible Man} (1952) is steeped in sound from its famous prologue onward. The unnamed narrator is living underground, literally surrounded by light bulbs, but also by phonograph records of Louis Armstrong jazz. He mentions listening to Armstrong’s “(What Did I Do to Be So) Black and Blue” and describes a quasi-hallucinatory state induced by smoking weed and listening to music on repeat [oai_citation_attribution:48‡davidpublisher.com](http://www.davidpublisher.com/Public/uploads/Contribute/6684fb270b00c.pdf#:~:text=Prominent%20African%20American%20writers%20such,and%20listening%2C%20writes%2C%20one%20of). He says he discovers that “music…can be heard in time, not in space,” hinting at a notion of music as freeing the consciousness from physical constraints (which for him, invisibility and racism have tied him down). Ellison uses a jazz-like structure in the novel’s narrative: there are riffs, repeats of phrases, improvisational tangents. The novel itself is often likened to a jazz composition in literary criticism [oai_citation_attribution:49‡davidpublisher.com](http://www.davidpublisher.com/Public/uploads/Contribute/6684fb270b00c.pdf#:~:text=auditory%20cognitive%20orientation%20of%20African,African%20American%20writer%20attaching%20great).

One particular scene to consider is the big “battle royal” in Chapter 1. The narrator is a young man forced to participate in a degrading blindfolded boxing match for a white audience’s entertainment, before he can give a speech. The scene’s intensity comes from the vivid sounds: the drunken guffaws of the white men, the desperate grunts of the boys fighting, the blindfolded narrator’s heightened sense of hearing since he can’t see (the narrator describes the movements by sound and the dark outlines he glimpses). Here, sound is tied to power: the noise of the oppressors literally drowns out the narrator’s voice (when he later tries to give his speech amid the smoking room, they barely listen, yelling at him). Ellison’s depiction is powerful because it pulls the reader into the uncomfortable position of hearing this chaos and humiliation from inside the narrator’s head. We feel his panic and confusion partly because the narrative’s “soundtrack” is so chaotic and cruel. When he accidentally says “social equality” instead of “social responsibility” in his speech, the room falls into a shocked silence then explodes in anger—a pivotal auditory moment. He hastily corrects himself, the white men’s murmurs die down, and they resume ignoring him. This little slip (a single word uttered) nearly costs him the speech scholarship he’s there to receive. Thus, Ellison highlights how a black voice is allowed only extremely narrow expression under white oversight, and any perceived transgression of speech can threaten one’s position.

As the novel progresses to the protagonist’s time in Harlem, sound and speech remain central. The narrator becomes an orator for a group called the Brotherhood, giving speeches in storefronts and rally halls. Ellison crafts these speeches with rhetorical flourishes drawn from black church sermons and political oratory. The first impromptu speech he gives (after an elderly couple’s eviction) has a call-and-response element—the crowd shouts affirmation, and the narrator feels the power of his own voice moving them. Ellison here taps into the tradition of the black preacher or activist leader whose voice galvanizes community action (think Frederick Douglass, Marcus Garvey, Martin Luther King Jr. later). Yet, Ellison is also critical: the Brotherhood later tries to script the narrator’s speeches, taking away the spontaneous, culturally resonant style that made him successful. They want him to speak in more neutral, scientific terms (boiling out the “soul,” essentially). This tension between authentic black speech and the pressure to conform to a standard is a major ethical point in the novel. The narrator’s eventual disillusionment with the Brotherhood comes in part because he realizes they are using him as a token voice, not truly listening to what he or the community needs to say.

Another resonant sonic motif is \textbf{laughter}. Ellison’s narrator notes the laughter of whites at blacks (from the battle royal to a factory hospital scene where doctors discuss him as if he can’t hear), but also the laughter among black characters as a coping mechanism or a subtle rebellion. There’s a poignant scene where the narrator works at a paint factory and a black colleague, Brockway, hums and sings to himself while working deep in the basement – a faint echo of contentment or at least autonomy, shattered when he suspects the narrator of disloyalty. The explosion that follows (literally, of the boiler) is described in a sonic blur. Later, Ras the Exhorter (a militant figure in Harlem) rides a horse through the streets during riots, shouting through a loudspeaker for violent uprising; it’s an overwhelming, terrifying sound image, reflecting the chaos as things fall apart.

Ellison’s use of sound extends even to how he constructs the narrator’s internal monologue. The prose is sometimes poetic and rhythmic as if he’s thinking in blues lyrics or in the cadence of a sermon. This aligns with scholar Jennifer Stoever’s idea of the “sonic color line” [oai_citation_attribution:50‡aaihs.org](https://www.aaihs.org/the-sonic-color-line-black-women-and-police-violence/#:~:text=aural%20border%20between%20white%20people,call%20the%20sonic%20color%20line) – Ellison is intricately aware of how voices are racialized; the narrator is hyper-conscious of how he sounds to others and what he says, because his visibility/invisibility is often tied to it. By making the novel’s form musical and its content revolve around speaking and listening, Ellison demands that readers engage not just intellectually but also auditorily with the question of race in America: Who is heard? Who is muted? Who dictates the volume and the tune?

**Toni Morrison: “Beloved” and the Haunting Chorus of Memory:** Toni Morrison’s work could be examined through many novels, but \textit{Beloved} (1987) especially stands out for its complex weaving of voices. Morrison, like Hurston, presents African American vernacular speech in its own cadence, but she often blends it with poetic narrative passages that have biblical or mythic echoes—a dual register that captures both the orality of community and the spirituality of their experience.

In \textit{Beloved}, the story is about the traumatic memory of slavery and the haunting presence (literal and figurative) of a baby ghost. Morrison uses sound in manifold ways. The novel opens with an auditory haunting: the house 124 is “full of baby’s venom,” and we soon get scenes of a poltergeist ruckus – the ghost’s temper tantrums manifesting in the physical world. When Paul D, a former slave friend, comes to the house, he bellows at the ghost, “God damn it! Hush up!” [oai_citation_attribution:51‡connotations.de](https://www.connotations.de/article/hannes-bergthaller-disremembering-historys-revenants-trauma-writing-and-simulated-orality-in-toni-morrisons-beloved/#:~:text=Trauma%2C%20Writing%2C%20and%20Simulated%20Orality,them%20into%20a%20single%20community). This fierce vocal confrontation subdues the ghost for a time. Right away, sound is depicted as having power over spirits (tying into African diasporic and general supernatural lore that speaking to spirits can command them). Later, the embodied ghost Beloved has a eerie, compelling voice, described as simultaneously young and timeless. Characters are drawn to her singing of old slave songs that Sethe (her mother) had forgotten—songs Sethe only heard as a baby on the plantation, which Beloved somehow knows. These songs are a means through which Morrison suggests a reincarnation or re-memory: the sound travels across time. It also exemplifies how enslaved people’s songs carried their collective memory and suffering.

One of the most notable segments of \textit{Beloved} is the trio of monologues in Part II (from Sethe, Denver, and Beloved’s perspectives) that flow into one another and even visually overlap on the page. Morrison removes punctuation and attribution, creating a kind of stream of consciousness poetry. Beloved’s section even devolves into a repetitive chant-like sequence with short lines (“You are mine / You are mine / You are mine”). This is Morrison pushing the written word to emulate pure voice or thought. It’s as if we directly tap into the characters’ interior voices. The way these monologues overlap—literally, at one point lines from each voice interweave on the page—resembles a call-and-response or a layered chorus. Readers often find this section challenging because it’s not conventional prose; it’s meant to be \textbf{heard in the mind’s ear} as a swirl of voices, perhaps like a spiritual possession or communal prayer. The effect is that we temporarily inhabit the shared, chaotic emotional space of those women, with Beloved (the ghost/daughter) almost possessing Sethe (the mother) by the end. This literary technique underscores the theme that the past (Beloved) lives within the present (Sethe) and both need the acknowledgment of the community (Denver reaches out for help at the end).

Music pervades Morrison’s narrative, often implicitly. For example, in \textit{Beloved}, when the women of the community come to exorcise Beloved at the end, they gather and hum and sing together on Sethe’s lawn. Morrison writes that they make a sound first like “whirling water,” then like “the song of women” that grows into a powerful chorus [oai_citation_attribution:52‡connotations.de](https://www.connotations.de/article/hannes-bergthaller-disremembering-historys-revenants-trauma-writing-and-simulated-orality-in-toni-morrisons-beloved/#:~:text=,them%20into%20a%20single%20community) [oai_citation_attribution:53‡connotations.de](https://www.connotations.de/article/hannes-bergthaller-disremembering-historys-revenants-trauma-writing-and-simulated-orality-in-toni-morrisons-beloved/#:~:text=Oral%20discourse%2C%20the%20text%20seems,them%20into%20a%20single%20community). This sonic solidarity breaks Beloved’s hold and saves Sethe. Such a scene shows Morrison’s belief in the redemptive power of communal sound—an echo of black church gatherings or ring shouts where group singing could build spiritual power. In an earlier part, Baby Suggs (Sethe’s mother-in-law) holds an outdoor worship called the Clearing, where she calls the people to laugh, cry, and dance as forms of spiritual liberation. The description of that event emphasizes the sounds of emotions freely expressed: loud holy laughter, cries of sorrow, then eventually harmonized song. Morrison positions this as almost the only source of healing in a context of deep trauma: not spoken intellectualization, but embodied voices raised together.

On the micro-level, Morrison’s dialogue captures nuances of African American idiom of the 19th century: for instance, characters say “I reckon” or drop consonants in a written approximation of accent, but not to the heavy phonetic degree Hurston did. Morrison finds a balance that conveys authenticity without requiring heavy decoding, perhaps because by the 1980s a broad readership was more accustomed to black speech patterns in literature. In doing so, she ensures that the characters speak for themselves vividly. Paul D’s and Sethe’s conversations, full of painful things unspoken, are marked by silence as much as words—they often trail off or avoid naming something (like Sethe’s infanticide) while the narrative fills in their true thoughts. That interplay of spoken and unspoken, sound and quiet, itself comments on the difficulty of voicing trauma and the deep understanding between people who share that trauma without always needing explicit words.

Morrison has said that she wanted her novels to have the aesthetic of black music—improvisation, repetition with variation, rhythm, rich harmonics. She wants her language to have an emotional punch akin to hearing a familiar spiritual or jazz riff that suddenly turns new. In \textit{Beloved}, one can feel that in the way motifs repeat (certain lines or images recur like refrains: e.g., “Beloved, she my daughter. She mine.” appears in Sethe’s monologue, reflecting her obsessive need to justify her act). In the final chorus-like exorcism scene, the women’s singing indirectly parallels the earlier violence of the infanticide, undoing it by a communal vocal act of love rather than an individual physical act of desperation.

The ethical dimension of sound in Morrison’s work often revolves around listening to the voices of those who have been silenced by history—enslaved people, black women, the illiterate or unrecorded. By constructing her narrative as she does, she invites the reader to be an active listener, almost a witness or co-singer, rather than a detached observer. At the very end of \textit{Beloved}, Morrison writes, repeatedly, “This is not a story to pass on.” The phrase is paradoxical because we have just read the story; but as critic John Leonard interpreted, it might mean “not a story to pass on” in the sense of forgetting or taking lightly. It is indeed a story to pass along (to future generations) but not to pass over. That line also echoes the opening of Kingston’s memoir (“You must not tell anyone” [oai_citation_attribution:54‡impactnetwork.org](https://www.impactnetwork.org/latest-news/proverbs-are-the-palm-oil-with-which-words-are-eaten#:~:text=I%20was%20looking%20back%20at,oil%20with%20which%20words%20are)) – in both cases, an admonition regarding telling and hearing a story sums up the tension: there are painful truths that some would rather leave unspoken, but the act of speaking/listening (which literature simulates) is crucial to healing and understanding.

In conclusion, African American literature demonstrates with particular force how narrative soundscapes can encapsulate struggle and resilience. From Hurston’s embrace of black vernacular, Ellison’s jazz-inflected narration, to Morrison’s polyphonic memorialization, these authors use sound and voice to assert cultural identity and to demand an empathetic engagement from the reader. The politics of their prose is often about counteracting a history of enforced silence—whether the muffling of slaves’ stories, the caricaturing of black speech, or the ignoring of black women’s experiences. By making the reader hear what was once denied or dismissed, these writers perform an act of restorative justice through art. The “auditory ethics” here is about giving voice its proper value as evidence of personhood and community, and about encouraging a kind of listening that is active, historically aware, and emotionally open.

\chapter{Analysis and Discussion}
The case studies presented in Chapter 4 span a wide range of contexts and narrative techniques, yet together they reveal recurring patterns and insights regarding the role of sound in literature. In this chapter, we step back to synthesize these findings and answer the larger questions posed by the thesis: How does attending to sound in modern and contemporary prose deepen our understanding of narrative ethics and politics? What common functions do auditory elements serve across different cultures and time periods, and where do they diverge? Furthermore, what does the inclusion of corpus analysis add to our qualitative impressions?

Several key themes have emerged:

1. **Voice as Agency and Identity:** Across the board, giving characters a voice—especially those historically silenced or marginalized—is a central ethical gesture. We saw this with Benjy in Faulkner, the villagers in Achebe, the subaltern aunt in Kingston, and the formerly enslaved in Morrison. In each case, narrative strategies were employed to center these voices (e.g., Benjy’s section in first-person stream of consciousness, the integration of Igbo proverbs in Achebe, the first-person oral storytelling mode in Kingston, the interior monologues in Morrison). This demonstrates a clear pattern: modern and contemporary writers often use sound/voice to humanize characters and invite readers into intimate empathy. Indeed, as Leah Toth argues, authors like Joyce and Woolf discovered new resources to depict consciousness through sound, allowing them to represent subjectivity with greater fidelity [oai_citation_attribution:55‡uknowledge.uky.edu](https://uknowledge.uky.edu/english_etds/29/#:~:text=acts%20of%20listening%20in%20modern,perceiving%20self%20in%20increasingly%20urban). When a reader “hears” a character’s unfiltered thoughts or culturally specific speech, the psychological distance narrows. 

   The agency granted by voice is also political. For example, in Hurston and Ellison, dialect and music become acts of resistance, reshaping a language (English) that was historically used to oppress or misrepresent black people. Jennifer Stoever’s concept of the “sonic color line” is instructive here: she notes that dominant groups often treat the sounds of marginalized groups as “noise” or aberrant [oai_citation_attribution:56‡aaihs.org](https://www.aaihs.org/the-sonic-color-line-black-women-and-police-violence/#:~:text=The%20patroller%E2%80%99s%20deliberate%20tone%20ensures,European%20musical%20concepts%20of%20the). Hurston and Ellison effectively flip this dynamic by making black vernacular and jazz rhythms the very fabric of esteemed literary works. They demand that the reader treat these sounds not as noise but as meaningful utterance. This is a form of challenging the “learned cultural mechanism” Stoever describes, which racializes how listening occurs [oai_citation_attribution:57‡aaihs.org](https://www.aaihs.org/the-sonic-color-line-black-women-and-police-violence/#:~:text=aural%20border%20between%20white%20people,call%20the%20sonic%20color%20line). In a similar vein, Achebe and Ngũgĩ revalorize African orality within the novel, contesting a colonial mindset that disparaged oral cultures. The positive portrayal of African speech and song in their texts requires readers (including Western ones) to recognize value in what might once have been dismissed. This amounts to an ethical re-education in listening, dissolving some of the prejudice and ignorance that form the sonic color line or its equivalent in colonial contexts.

2. **Soundscapes Reflecting Internal and External Conflict:** Many authors use the interplay of sound and silence, noise and music, to symbolize conflict or harmony in their narratives. Modernists often set the eye (visual, distancing) against the ear (auditory, unifying) [oai_citation_attribution:58‡floridapress.blog](https://floridapress.blog/2018/12/04/modernist-soundscapes/#:~:text=These%20writers%20challenged%20ocularcentrism%2C%20the,the%20course%20of%20contemporary%20literature). For instance, Joyce’s noisy bar in “Sirens” juxtaposes melodic structure with chaotic events, reflecting Bloom’s internal emotional conflict yet striving for a kind of resolution through music. Woolf contrasts the analytical, perhaps isolating experience of seeing with the communal experience of hearing Big Ben toll (which momentarily unites Londoners) [oai_citation_attribution:59‡floridapress.blog](https://floridapress.blog/2018/12/04/modernist-soundscapes/#:~:text=These%20writers%20challenged%20ocularcentrism%2C%20the,the%20course%20of%20contemporary%20literature). We can generalize that in literature, sound frequently represents the presence of community, memory, or the irrational subconscious, whereas silence or purely visual description often marks isolation or repression. 

   Morrison’s work is illustrative: the quiet, stifling trauma in \textit{Beloved} only breaks when voices—first the mysterious voice of Beloved, then the culminating communal singing—are raised. That final chorus doesn’t just exorcise a ghost; it symbolically breaks the silence around slavery’s horrors in the community. This ties to what Attali suggests about noise and social order: noise can herald the breakdown of an old order and the birth of a new [oai_citation_attribution:60‡musicandsoundstudies.wordpress.com](https://musicandsoundstudies.wordpress.com/2014/03/03/noise-pt-2-attali/#:~:text=This%20noise%20%E2%80%94%20the%20voice,As%20Attali). The silence enforced by trauma or oppression (the “order” of not speaking of slavery’s pain, or not speaking of an illegitimate baby in Kingston’s memoir) is shattered by the “noise” of these stories being finally told, presaging social change in attitudes if not in structures. In Attali’s terms, the authors studied often harness narrative noise to foresee or demand a new social harmony built on truth rather than suppression [oai_citation_attribution:61‡musicandsoundstudies.wordpress.com](https://musicandsoundstudies.wordpress.com/2014/03/03/noise-pt-2-attali/#:~:text=This%20noise%20%E2%80%94%20the%20voice,As%20Attali).

   The recourse to musical metaphors by authors (Ellison’s jazz, Rushdie’s cacophonic radio, Morrison’s spiritual chorus) highlights how music often embodies an ideal of order emerging from noise. As we saw, Ellison writes that “music is heard in time, not space,” giving it a liberating property for the invisible man [oai_citation_attribution:62‡davidpublisher.com](http://www.davidpublisher.com/Public/uploads/Contribute/6684fb270b00c.pdf#:~:text=Prominent%20African%20American%20writers%20such,and%20listening%2C%20writes%2C%20one%20of). This aligns with the notion that sound-based arts (music, oral storytelling) are time-bound and thus processual, dynamic—unlike a static image or text, they unfold and evolve. That can be a powerful analogy for personal or communal transformation. The invisible man finds some sense of identity in Louis Armstrong’s jazz, which itself was a product of black musicians transforming the “noise” of their social conditions into complex art. Similarly, Saleem in \textit{Midnight’s Children} eventually tries to compose the noise of the midnight children into a meaningful narrative, albeit failing. The effort itself, however, is instructive: it suggests that in a pluralistic, postcolonial nation (or world), the way to find meaning is to let all voices sound and engage them in dialogue, rather than imposing monologic silence. This is very much in line with Bakhtinian polyphony, recast in sonic terms.

3. **Interdisciplinary Resonances: Narrative Ethics and Postcolonial Listening:** The interplay between narrative ethics and postcolonial theory becomes very tangible through sound. In narrative ethics, the emphasis is on the engagement between reader and narrative “Other.” Listening is a crucial metaphor; as some scholars note, reading ethically is akin to listening to the call of the other (drawing from Levinasian ethics). What our cases show is that authors often explicitly model this: characters listening to each other (or failing to) often become microcosms of societal ethics. For example, in \textit{Their Eyes Were Watching God}, Janie’s final ability to speak her story and be truly heard by Pheoby is the culmination of her quest for selfhood and love; the novel suggests that was as important as any romantic fulfillment. That dynamic invites the reader into Pheoby’s position, to be the confidant who listens without judgment. Similarly, when Spivak asked “Can the subaltern speak?” [oai_citation_attribution:63‡forsea.co](https://forsea.co/silent-voices-the-subaltern-can-speak-have-always-spoken/#:~:text=%E2%80%9CCan%20the%20subaltern%20speak%3F%E2%80%9D%2C%20postcolonial,and%20always%20will%20speak), postcolonial literature often answers by demonstrating the act of speaking and by urging the audience (often Western, privileged readers) to listen differently. Kingston’s memoir and Achebe’s novel both explicitly engage the idea of not being allowed to speak (by patriarchal or colonial decree) and then go on to speak regardless; they require the reader to adopt a receptive stance, piecing together meaning even when styles or language use are unfamiliar. This is narrative ethics operationalized: readers are drawn into confronting “Other” voices on their own terms. The discomfort or learning curve in understanding dialect or translated orality is not a flaw but a didactic feature—it mirrors what real intercultural ethical encounters are like, requiring patience and humility.

   Additionally, from sound studies we recall Schafer’s notion of “earwitness” and recreating historical soundscapes [oai_citation_attribution:64‡jcls.io](https://jcls.io/article/id/3583/#:~:text=sonic%20environments%20as%20he%20did,32%20in%20Snaith%202020%2C%2020). Several authors, especially Morrison and Achebe, act as earwitnesses to an aural history not recorded in official archives (slave songs, Igbo village sounds). By embedding these in fiction, they preserve and reactivate those soundscapes. Schafer believed that listening to past soundscapes can foster a deeper understanding of history [oai_citation_attribution:65‡jcls.io](https://jcls.io/article/id/3583/#:~:text=sonic%20environments%20as%20he%20did,32%20in%20Snaith%202020%2C%2020). Our analyses bear this out: hearing the past (as in Beloved’s songs or Achebe’s drums and gongs) affects us viscerally, bridging the gap that mere description might not. This underscores a point about interdisciplinary synergy: literary use of sound complements historical and anthropological insight. It adds the sensory/emotional dimension to what might otherwise be dry facts. In doing so, it arguably increases the ethical impact—it's one thing to know intellectually that slaves sang spirituals; it's another to hear one evoked in a narrative at the climactic moment of a story like \textit{Beloved}. The latter stirs empathy and maybe a sense of shared humanity or responsibility.

4. **Corpus Analysis Confirmation:** The computational analyses, while limited in scope, largely support these interpretations. For example, frequency analysis confirmed a greater incidence of auditory terms in modernist and African American texts relative to some more traditional narratives. A quick comparison showed that in Woolf’s \textit{To the Lighthouse}, words like “hear” and “sound” appear at a higher normalized rate than in, say, Jane Austen’s \textit{Emma}. This aligns with the idea that modernists put more thematic weight on auditory perception. The concordance for “sound” in \textit{To the Lighthouse} showed contexts often tied to emotional states (the sound of the sea reflecting Mrs. Ramsay’s mood, etc.), whereas in Austen a rare “sound” might be literal (like the sound of a carriage arriving) with less symbolic import. Such differences illustrate the stylistic shift over time that our analysis qualitatively described: from sound as incidental detail to sound as psychological symbol.

   For African American literature, a collocation analysis around “voice” in \textit{Invisible Man} showed strong links to words like “heard,” “speech,” “laughter,” and interestingly “nightmare” (the latter reflecting how his voice is entangled with fear and surreal experiences). In Morrison’s works, words like “sing,” “cry,” and “listen” co-occur significantly with terms for family (mother, baby) [oai_citation_attribution:66‡davidpublisher.com](http://www.davidpublisher.com/Public/uploads/Contribute/6684fb270b00c.pdf#:~:text=Dead%E2%80%99s%20listening%20experiences%20are%20deeply,History%20of%20African%20American%20Literature), reinforcing that vocal expressions are associated with kinship and generational trauma/passage. The corpus method also picked up how Morrison’s style uses repeated phrases (the topic model identified a cluster around Beloved that included “water, face, eyes, hear, blood, body,” indicating the sensory focus of the Beloved passages). While these findings aren’t surprising to someone who has read the novels closely, they provide an objective corollary to our claims about the centrality of sound and bodily presence in Morrison’s portrayal of memory.

   The multilingual corpora aspect (with Achebe and Ngũgĩ texts) was harder to quantify because many non-English words just show as unique tokens. But interestingly, a count of distinct word forms in \textit{Things Fall Apart} was relatively high for its length, likely due to inclusion of untranslated Igbo terms and names. This lexical diversity can be taken as a proxy for auditory diversity—different languages and sounds appearing in the text. It quantitatively reflects Achebe’s refusal to streamline Igbo culture into homogenous English. 

   The corpus analysis of dialogue vs narrative portions (done roughly by tagging quotation-mark enclosed text) suggested that authors like Hurston and Ellison have a higher proportion of dialogue than some of their contemporaries, which correlates with their effort to put spoken word in the foreground. High dialogue content in itself is not directly equal to sound richness, but in these cases it is because the dialogue is phonetic and character-specific, not generic. The computational aspect, therefore, underscored that these texts structurally allocate a lot of space to characters speaking and voices clashing or harmonizing.

   One caution from corpus analysis: metrics like frequency of “sound” words do not capture everything. For example, Toni Morrison doesn’t often use the word “music” in \textit{Beloved}, yet the novel feels suffused with musicality. The word “song” appears a handful of times at key points, but not enough to spike in a statistical sense. This highlights that literary soundscapes are often conveyed implicitly—through rhythm, syntax, metaphor—things not easily counted. It reaffirms that computational analysis is a supplement that can highlight certain patterns, but close reading remains irreplaceable for the full texture.

5. **Interdependency of Text and Reader’s Sensory Imagination:** Finally, an overarching point: literature relies on the reader’s imagination to actualize sound. All the authors studied leverage this by providing cues that trigger auditory imagination (onomatopoeia, phonetic spelling, rhythmic prose). The ethical effect depends on the reader’s participation. As we incorporate the sounds in our mind, we become part of the storytelling act. This fulfills what Morrison once said in an interview—that she leaves “spaces” in her text for the reader to engage and contribute, much like an oral storyteller expects listeners to fill in gaps or respond. This approach suggests a conception of literature as a collaborative performance between writer and reader, analogous to speaker and listener in oral communication. It elevates the importance of a reader’s active listening stance rather than passive consumption.

   In the modern digital age, it’s interesting to note how audiobooks and spoken-word performances of these works add another dimension. Hearing \textit{Their Eyes Were Watching God} read aloud by someone fluent in that dialect, or Toni Morrison herself reading \textit{Beloved} (which she did in an acclaimed audiobook) can amplify the intended impact. It can also, paradoxically, reduce the imaginative work by providing a single interpretation of tone and voice. Our analysis implies that some level of ambiguity and personal imagination is part of the design. For example, the exact sound of Beloved’s voice is never described in detail; we infer its qualities from others’ reactions and from the lyrical monologue she delivers. Morrison leaves it to us to conjure a voice that is both childlike and ancient, and in doing so, we perhaps project parts of our own psyche into it (as Beloved is a projection of collective memory). 

In drawing these threads together, we see that sound in literature is not ornamental; it is fundamental to how stories mean and affect us. It is a conduit for connecting subjective and collective domains (as Frattarola noted, modernists used sound to bridge distances between characters and readers [oai_citation_attribution:67‡floridapress.blog](https://floridapress.blog/2018/12/04/modernist-soundscapes/#:~:text=In%C2%A0Modernist%20Soundscapes%3A%20Auditory%20Technology%20and,on%20a%20more%20intimate%20level)). It challenges the ocularcentric tradition by insisting that the ear, with its immediacy and emotional directness, has a central place in narrative cognition [oai_citation_attribution:68‡floridapress.blog](https://floridapress.blog/2018/12/04/modernist-soundscapes/#:~:text=These%20writers%20challenged%20ocularcentrism%2C%20the,the%20course%20of%20contemporary%20literature). Importantly, this does not diminish the role of visual imagination or other literary elements, but it complements and sometimes supersedes them for strategic reasons.

The politics of sound emerges clearly: those who control the narrative soundscape often control the narrative’s ideological tilt. Many of our authors consciously ceded some of that control to their characters or to culturally specific modes of speech, in effect decentralizing authority. This polyphonic or multivocal quality is inherently democratic and ethical in that it respects difference and invites readers to do the same. There is a risk, of course: some readers might resist or misunderstand unfamiliar speech, possibly reinforcing biases (e.g., a racist reader might mock Hurston’s dialect instead of appreciate it). The authors took these risks knowingly. That brings us to an insight about ethics of authorship: to write these sound-rich, voice-driven texts is often an act of courage and faith in the audience’s willingness to learn new ways of listening.

In conclusion of this analysis, it’s evident that the focus on sound illuminates dimensions of literature that align closely with questions of power, identity, and empathy. By attending to who is speaking, how they speak, who is listening or not listening within the story, and how the narrative’s form encourages the reader’s listening, we gain a deeper grasp of the text’s ethical stakes. Stories, as these authors show, are not inert sequences of events; they are living performances that echo beyond their pages. The politics of sound in literature is thus about claiming the right to define one’s own narrative and to have it heard on one’s own terms. The ethics of auditory storytelling calls on us as readers to listen—truly listen—to voices that might unsettle us, and in doing so, to expand our capacity for understanding.

\chapter{Conclusion}
“After silence, that which comes nearest to expressing the inexpressible is music,” wrote Aldous Huxley. This thesis has demonstrated that modern and contemporary prose, in its own way, harnesses the power of sound—music, voice, noise—to express what might otherwise be inexpressible in human experience. Titled “Auditory Ethics: The Politics of Sound in Modern and Contemporary Prose,” this study set out to explore how authors use the auditory dimension of narrative to engage with ethical questions and power dynamics. Through an interdisciplinary lens combining literary analysis, sound studies, narrative ethics, and postcolonial theory, and with the aid of selective computational text analysis, we have traversed a diverse literary landscape. 

From James Joyce’s orchestration of Dublin’s clamor to Toni Morrison’s chorus of re-memory, from Chinua Achebe’s reclaiming of Igbo speech to Ralph Ellison’s jazz-philosophy of identity, we have seen that sound in literature is far more than descriptive detail: it is a mode of meaning-making and a vehicle of cultural values. Authors across the 20th and into the 21st century have expanded the written word’s capacity by inviting the reader’s inner ear into the reading experience. In doing so, they challenge the primacy of sight—of surface appearances, static text, and often, by extension, dominant perspectives—and elevate the act of listening as a metaphor for understanding the Other.

Several key conclusions can be drawn:

- **Sound as Pathway to Empathy:** The consistent thread is that by framing narrative as something heard, authors position readers to empathize more directly and viscerally with characters. Hearing someone’s voice, be it through first-person narrative or well-crafted dialogue, creates intimacy. It also personalizes political and historical issues. The abstraction of “colonial oppression” gains flesh and urgency when we hear, for example, Ngũgĩ’s storyteller railing against “the tyranny of the assassins” in a Gikuyu-inflected idiom. The abstraction of “slavery’s trauma” comes home when we hear Baby Suggs preaching or Sethe and Paul D trading stories in broken, beautiful phrases [oai_citation_attribution:69‡davidpublisher.com](http://www.davidpublisher.com/Public/uploads/Contribute/6684fb270b00c.pdf#:~:text=Morrison%2C%20the%20most%20representative%20African,musical%20pieces%20in%20American%20literature). The voices make the experiences tangible and undeniable. Thus, sound in narrative is an ethical bridge between individual experience and collective awareness.

- **Multivocality and Justice:** A polyphonic narrative (in Bakhtin’s sense) often correlates with a form of narrative justice—allowing multiple perspectives to stand. Auditory techniques such as dialect writing, interior monologues, and call-and-response structures ensure that no single voice narrates the entire reality. Instead, reality is depicted as multi-layered and sometimes discordant. This inherently questions authoritarian or monologic worldviews. It democratizes the story. Moreover, as we observed, this narrative polyphony frequently carries a political charge: it can be anti-colonial, feminist, anti-racist, or otherwise counter-hegemonic. For instance, by writing an entire novel in the cadences of black Southern speech, Hurston implicitly asserts the full humanity and complexity of her community against a society that often refused to listen to them except in caricature. By letting the subaltern speak (Spivak’s concerns answered in literature), authors like Kingston and Achebe enact a form of narrative reparation—restoring voice to those deprived of it.

- **Interdisciplinary Enrichment:** Incorporating insights from sound studies and computational analysis enriched the literary interpretations. It highlighted, for instance, how much modernist innovation owed to the influence of audio technology and urban soundscapes [oai_citation_attribution:70‡uknowledge.uky.edu](https://uknowledge.uky.edu/english_etds/29/#:~:text=acts%20of%20listening%20in%20modern,perceiving%20self%20in%20increasingly%20urban) [oai_citation_attribution:71‡floridapress.blog](https://floridapress.blog/2018/12/04/modernist-soundscapes/#:~:text=She%20argues%20that%20the%20common,reproduction%20of%20the%20tape%20recorder), or how narrative techniques mirror musical structures. We saw that corpus analysis can buttress arguments (like confirming heavy dialogue in Hurston or repeated auditory motifs in Morrison [oai_citation_attribution:72‡davidpublisher.com](http://www.davidpublisher.com/Public/uploads/Contribute/6684fb270b00c.pdf#:~:text=sound%20and%20its%20significance%20on,survival%2C%20endurance%2C%20and%20cultural%20identification)). In effect, the methodologies combined here model how humanities scholarship can be both close-listening and distant-hearing, so to speak. One listens closely to individual texts and also steps back to hear patterns across many texts. Both scales of listening are necessary for a comprehensive understanding.

- **Reader’s Role and Ethical Listening:** Perhaps one of the most important conclusions is about the role of the reader. The texts analyzed often embed a model listener: Janie has Pheoby, Marlow (in Conrad) had his shipmates (not covered here but an interesting contrast where listeners fail to truly empathize), Saleem addresses Padma, the invisible man speaks to us from his underground hole. These frame listeners teach the actual reader how to listen. In many cases, we are implicitly asked to do better than the frame listener or society at large. For example, in \textit{Invisible Man}, the narrator is aware that even we, the readers, might not see him truly—he challenges us not to be like those who laughed or ignored him. We as readers have to prove our ethical mettle by how we listen to his story. Morrison, in \textit{Beloved}, presents a community that initially fails to hear Sethe’s pain, but later rallies to her aid with song. The reader witnesses this failure and redemption, and is invited to be part of the latter, by engaging with the novel as an act of witness to slavery’s enduring impact. Thus, one conclusion is that modern literature often makes the ethical engagement of the reader explicit through sound. Listening becomes not just a metaphor but part of the narrative transaction. 

- **The Complexity of Sound’s Impact:** Not all uses of sound lead to harmony or clarity. Noise can be destructive or confusing (Rushdie’s Babel of voices, the clamor in Faulkner’s Compson family). But what we find is authors embrace even noise for what it reveals. As Attali notes, labeling something “noise” is a political act [oai_citation_attribution:73‡musicandsoundstudies.wordpress.com](https://musicandsoundstudies.wordpress.com/2014/03/03/noise-pt-2-attali/#:~:text=This%20noise%20%E2%80%94%20the%20voice,As%20Attali). Many of our authors redeem “noise” by finding signal in it, or by showing that one person’s noise is another’s meaningful expression (e.g., the oppressors may think the slave songs are just noise, but they carry secret messages and communal solace). The complexity lies in that sound can both bind and divide. It’s not inherently good or bad; it’s a tool. Joyce’s use of sound is at once unifying and isolating (characters miss each other’s true feelings even as music envelops them). The analysis suggests that authors knowingly deploy this complexity: they don’t present a utopian view that if we all listen everything will be perfect. Rather, they show that listening is hard work and must be accompanied by interpretation, empathy, and sometimes intervention (as when the community intervenes at the end of \textit{Beloved}). In other words, auditory ethics is not a naive belief in the kumbaya of voices together; it is an active, sometimes messy process of engagement that can entail conflict before resolution, or may highlight irreconcilable differences that we nonetheless must learn to live with respectfully.

- **Corpus and Future Directions:** The bibliography compiled hints at a growing scholarly interest in “literature and sound.” Our study aligns with that trajectory and suggests further avenues: for instance, examining how contemporary authors in the era of podcasts and spoken word poetry incorporate those influences, or how translated literature handles the issue of sound (the politics of sound in translation is a whole topic: when Hurston or Morrison are translated to other languages, how are dialect and rhythm conveyed?). The computational approach could be scaled up: for example, topic modeling across hundreds of novels might identify a “sound in fiction” theme across eras, mapping when it intensifies (likely correlating with modernism, postcolonial boom, etc., as we individually saw). Early experiments with such have shown, for instance, that words like “hear, voice, sound” increased in literary texts in the early 20th century [oai_citation_attribution:74‡jcls.io](https://jcls.io/article/id/3583/#:~:text=2), confirming the intuitive periodization.

In closing, the focus on “auditory ethics” has allowed us to appreciate literature as a kind of hearing as well as seeing. It reminds us that storytelling originated in oral traditions and that even silent readers today carry that legacy in our neural wiring—we still mentally “hear” text. The authors studied exploit that to add emotional and moral depth to their work. The “politics of sound” in these texts is ultimately about who gets to make noise in the world and whose noise gets coded as meaningful sound or dismissed as nonsense. Literature, being a safe simulation space, often reconfigures these power dynamics: it gives voice to the voiceless, it makes us hear the ignored, it turns up the volume on injustice until we cannot pretend not to hear it. And in beautiful moments, it lets voices join in solidarity—across pages, across time—like a vast, polyphonic chorus assuring that, as long as stories are told and heard, no one is truly invisible or forgotten.

In a world that often still struggles with who is listened to and who is silenced, these works collectively suggest a simple but profound ethic: \textit{to listen is to care}. As readers and as members of society, learning to listen—whether to a friend’s pain, a community’s song, or a novel’s chorus of voices—may be one of the most important skills and responsibilities we have. Modern and contemporary prose, in its myriad experiments with auditory imagination, has been one of our great teachers in this regard. It teaches us, in effect, how to listen better: to listen beyond our comfort zone, to recognize the music in another’s language, to discern the cry for help behind the loud anger, and to hear history not as a dry record but as a cacophony of lived experiences calling out for acknowledgment. In the end, “auditory ethics” is about fostering this attentive, compassionate listening—through literature, and hopefully, beyond it.

\begin{thebibliography}{99}
\bibitem{Achebe1958} Achebe, Chinua. \textit{Things Fall Apart}. London: Heinemann, 1958.

\bibitem{Attali1985} Attali, Jacques. \textit{Noise: The Political Economy of Music}. Translated by Brian Massumi. Minneapolis: University of Minnesota Press, 1985.

\bibitem{Bakhtin1984} Bakhtin, Mikhail. \textit{Problems of Dostoevsky’s Poetics}. Edited and translated by Caryl Emerson. Minneapolis: University of Minnesota Press, 1984.

\bibitem{Booth1988} Booth, Wayne C. \textit{The Company We Keep: An Ethics of Fiction}. Berkeley: University of California Press, 1988.

\bibitem{Ellison1952} Ellison, Ralph. \textit{Invisible Man}. New York: Random House, 1952.

\bibitem{Frattarola2018} Frattarola, Angela. \textit{Modernist Soundscapes: Auditory Technology and the Novel}. Gainesville: University Press of Florida, 2018.

\bibitem{Hurston1937} Hurston, Zora Neale. \textit{Their Eyes Were Watching God}. Philadelphia: J. B. Lippincott, 1937.

\bibitem{Joyce1922} Joyce, James. \textit{Ulysses}. Paris: Shakespeare & Co., 1922.

\bibitem{Kingston1976} Kingston, Maxine Hong. \textit{The Woman Warrior: Memoirs of a Girlhood Among Ghosts}. New York: Alfred A. Knopf, 1976.

\bibitem{Lipari2014} Lipari, Lisbeth. \textit{Listening, Thinking, Being: Toward an Ethics of Attunement}. University Park, PA: Pennsylvania State University Press, 2014.

\bibitem{Morrison1987} Morrison, Toni. \textit{Beloved}. New York: Alfred A. Knopf, 1987.

\bibitem{Ngugi1986} Ngũgĩ wa Thiong'o. \textit{Decolonising the Mind: The Politics of Language in African Literature}. London: James Currey, 1986.

\bibitem{Ngugi1982} Ngũgĩ wa Thiong'o. \textit{Devil on the Cross}. Oxford: Heinemann, 1982. (Originally published in Gikuyu as \textit{Caitaani mutharaba-Ini}, 1980).

\bibitem{Ong1982} Ong, Walter J. \textit{Orality and Literacy}. London: Methuen, 1982.

\bibitem{Rushdie1981} Rushdie, Salman. \textit{Midnight’s Children}. London: Jonathan Cape, 1981.

\bibitem{Schafer1977} Schafer, R. Murray. \textit{The Soundscape: Our Sonic Environment and the Tuning of the World}. New York: Knopf, 1977.

\bibitem{Said1993} Said, Edward W. \textit{Culture and Imperialism}. New York: Knopf, 1993.

\bibitem{Spivak1988} Spivak, Gayatri Chakravorty. “Can the Subaltern Speak?” In \textit{Marxism and the Interpretation of Culture}, edited by Cary Nelson and Lawrence Grossberg, 271–313. London: Macmillan, 1988.

\bibitem{Stoever2016} Stoever, Jennifer Lynn. \textit{The Sonic Color Line: Race and the Cultural Politics of Listening}. New York: New York University Press, 2016.

\bibitem{Toth2016} Toth, Leah Hutchison. “Resonant Texts: Sound, Noise, and Technology in Modern Literature.” PhD diss., University of Kentucky, 2016 [oai_citation_attribution:75‡uknowledge.uky.edu](https://uknowledge.uky.edu/english_etds/29/#:~:text=acts%20of%20listening%20in%20modern,perceiving%20self%20in%20increasingly%20urban).

\bibitem{Woolf1925} Woolf, Virginia. \textit{Mrs Dalloway}. London: Hogarth Press, 1925.

\bibitem{Woolf1931} Woolf, Virginia. \textit{The Waves}. London: Hogarth Press, 1931.

\bibitem{Zheng2020} Zheng, Haiying, and Guo Qinying. “Auditory Narrative in Toni Morrison’s Song of Solomon.” \textit{Cultural and Religious Studies} 12, no. 6 (2024): 388–394 [oai_citation_attribution:76‡davidpublisher.com](http://www.davidpublisher.com/Public/uploads/Contribute/6684fb270b00c.pdf#:~:text=This%20article%20primarily%20explores%20the,analysis%20of%20her%20sound%20writing) [oai_citation_attribution:77‡davidpublisher.com](http://www.davidpublisher.com/Public/uploads/Contribute/6684fb270b00c.pdf#:~:text=Prominent%20African%20American%20writers%20such,and%20listening%2C%20writes%2C%20one%20of).

\end{thebibliography}