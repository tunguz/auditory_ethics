\documentclass[12pt]{report}
\usepackage[utf8]{inputenc}
\usepackage{setspace}
\usepackage{csquotes}
\usepackage{chicago}
\usepackage{hyperref} % Add hyperref package

\setlength{\parindent}{1.5em}
\setlength{\parskip}{0.5em}
\doublespacing

\begin{document}

\title{\textbf{Auditory Ethics: The Politics of Sound in Modern and Contemporary Prose}}
\author{[Author Name]}
\date{Ph.D. Thesis \\ \today}
\maketitle

\tableofcontents

\chapter*{Abstract}
\addcontentsline{toc}{chapter}{Abstract}
This thesis investigates the intersection of sound and ethics in twentieth- and twenty-first-century prose, examining how modern and contemporary authors incorporate auditory experience into narrative form and content. It expands the scope of analysis beyond canonical figures such as James Joyce, Virginia Woolf, Toni Morrison, and Maxine Hong Kingston to include a wider range of modernist, postcolonial, and postmodern writers. Drawing on interdisciplinary theoretical perspectives from sound studies, narrative ethics, postcolonial theory, and literary modernism/postmodernism, the project develops the concept of “auditory ethics” to describe how literature engages the politics of sound—who is heard, who is silenced, and how acts of listening are represented. The methodology combines close reading with corpus linguistics and computational text analysis, using digital tools to complement qualitative interpretation. Through case studies spanning high modernist experiments with interior monologue, postcolonial narratives reclaiming oral traditions, African American literature’s engagement with music and voice, and postmodern fiction’s confrontation with noise and media, the thesis demonstrates that sound in literature is never neutral. Rather, literary sounds and voices carry ethical weight and political implications, challenging readers to become active listeners. The study finds that authors across diverse contexts use auditory techniques to bridge distances between characters, critique power dynamics, and invite empathetic engagement. By foregrounding the auditory dimension of prose, these writers transform reading into an ethical act of listening. The thesis concludes that attending to the “auditory ethics” of literature enriches our understanding of narrative form and its social resonances. An extensive bibliography is provided in Chicago style. 

\chapter{Introduction}
Sound is fundamental to human experience, yet literature—an ostensibly silent, written medium—has continually sought to evoke and harness the auditory realm. This dissertation examines how modern and contemporary prose writers use sound as a narrative and ethical resource. Titled \textit{“Auditory Ethics: The Politics of Sound in Modern and Contemporary Prose,”} the study explores the intersection of auditory experience with questions of ethics and power in fiction from roughly the early twentieth century to the present. It asks: In what ways do novelists and storytellers incorporate the senses of hearing and sound into their works, and to what effect? How do literary representations of sound (from spoken voices and dialogue to musical structures, noise, silence, and the rhythms of language itself) engage with ethical questions about empathy, otherness, and justice? And how do these auditory elements reflect or challenge the politics of who gets to speak and who is heard in society?

The phrase “auditory ethics” in the title signifies the dissertation’s central claim: that paying attention to sound in literature—both the sounds depicted within stories and the sound-like qualities of the narrative voice or language—reveals an ethical dimension of storytelling. Many writers implicitly treat listening as an ethical act, one that can bridge distances between people or highlight injustices when certain voices are ignored. The “politics of sound” refers to how power relations play out through sound: whose voices dominate or are marginalized, how noise can be used to oppress or to resist, and how modern technological and cultural shifts in the auditory domain influence literature. By analyzing prose through this lens, the study uncovers the strategies authors use to politically charge the act of listening and the representation of sound.

This research is rooted in literary analysis but draws upon multiple theoretical frameworks. It engages with \textbf{sound studies}, an interdisciplinary field that examines sound’s cultural, technological, and philosophical aspects. Insights from sound studies help illuminate how literature responds to what has been called the “sonic turn” in modern culture, when advancements like the phonograph, radio, and telephone changed how people experienced sound. \href{https://uknowledge.uky.edu/english_etds/29/#:~:text=acts%20of%20listening%20in%20modern,perceiving%20self%20in%20increasingly%20urban}{(uknowledge.uky.edu)}. \href{https://floridapress.blog/2018/12/04/modernist-soundscapes/#:~:text=She%20argues%20that%20the%20common,reproduction%20of%20the%20tape%20recorder}{(floridapress.blog)}. The project also employs concepts from \textbf{narrative ethics}, which is concerned with the moral relations between storytellers, characters, and readers. Narrative ethics provides tools to discuss how giving a character a voice or asking the reader to “listen” to a text can be seen as ethical or unethical. In addition, the thesis incorporates \textbf{postcolonial theory} to address literature emerging from colonial and formerly colonized contexts, where issues of voice, silencing, and the recovery of oral traditions are paramount. Finally, it situates its discussion in the literary-historical contexts of \textbf{modernism and postmodernism}, two movements that, in different ways, revolutionized artistic form—including experiments with sound and language in prose.

A key goal of this study is to avoid privileging any single theoretical framework at the expense of others. Instead, it demonstrates how these perspectives can complement one another. For example, a close reading of a novel by \textit{Virginia Woolf} can be enriched by sound studies (noting her use of onomatopoeia and rhythm), by narrative ethics (considering how her narrative technique invites empathetic listening to characters’ inner lives), and by a postcolonial awareness (if examining how voices of different social backgrounds appear in her work). In treating these frameworks even-handedly, the thesis shows that the politics of sound in literature is a multifaceted subject: it is at once formal/aesthetic, ethical, and sociopolitical.

The corpus of literature examined is purposefully broad. While the initial inspiration comes from authors mentioned in the abstract—figures like James Joyce, Woolf, Toni Morrison, and Maxine Hong Kingston, all of whom explicitly concern themselves with voice and sound—this study extends to additional writers across the anglophone world. High modernist authors such as \textit{Dorothy Richardson}, \textit{William Faulkner}, and \textit{Samuel Beckett} feature in the discussion alongside Joyce and Woolf, illustrating how early twentieth-century prose experimented with auditory techniques. Postcolonial and transnational voices including \textit{Chinua Achebe}, \textit{Ngũgĩ wa Thiong’o}, and \textit{Salman Rushdie}, as well as Asian American and diasporic writers like Kingston, broaden the scope to colonial and immigrant contexts where reclaiming voice is often a central theme. African American literature receives dedicated attention through the works of \textit{Ralph Ellison}, \textit{Zora Neale Hurston}, and Morrison, highlighting the legacy of oral culture and music (from folktales to jazz and blues) in shaping narrative form and ethics. Finally, late twentieth-century and postmodern writers such as \textit{Don DeLillo} and \textit{Thomas Pynchon} are considered to show how the theme of sound (and conversely, noise) evolves in an era of information saturation and media networks. By examining this wide array of authors, the thesis underscores that the role of sound in literature is not confined to any one period or genre—it is a pervasive and evolving concern.

Methodologically, this project combines traditional literary scholarship with digital humanities approaches. Close reading remains the primary mode of analysis: careful attention is given to how specific texts represent sound and listening. For instance, what words, images, or narrative structures do authors use to evoke sound? How do they describe acts of listening or the absence of sound (silence)? How is dialogue rendered on the page, and does it have qualities that suggest particular accents, rhythms, or musicality? Such questions are addressed through detailed textual analysis in the case study chapters. Alongside this, the thesis employs \textbf{corpus linguistics and computational text analysis} as a supplementary approach. Using a curated digital corpus of literary texts, the study quantifies certain patterns—for example, the frequency of sound-related words or onomatopoeic expressions in modernist vs. Victorian novels, or the prevalence of phrases like “listen” and “hear” in narratives of different eras. These quantitative findings provide empirical support and context for the interpretive claims, although they remain subservient to the qualitative analysis. In effect, the digital methods function as a magnifying lens, bringing into focus broad trends that might escape notice in isolated readings. For example, a computational scan can reveal that authors like Joyce and Morrison use an unusually high density of auditory imagery compared to their contemporaries, reinforcing the notion that sound is central to their art. Such data, when used carefully, enriches the argument without overshadowing the nuanced reading of individual works.

The structure of the dissertation reflects a conventional progression from theory and context to application and synthesis. Chapter 2 offers a review of relevant literature and theoretical background, surveying work in sound studies, narrative theory, postcolonial studies, and related fields to establish the scholarly landscape. Chapter 3 discusses the methodology in detail, explaining how close reading and computational analysis are conducted and justified. Chapter 4 presents a series of case studies grouped by theme and historical context, each examining selected authors and texts to illuminate different facets of auditory ethics: from modernist “soundscapes” to postcolonial orature, from the sonic ethos of African American storytelling to the metafictional play with noise in postmodernism. Chapter 5 provides an analysis and discussion that draws connections among the case studies, synthesizing findings and linking back to the theoretical frameworks. Finally, Chapter 6 concludes the study, summarizing the insights gained and reflecting on their implications for literary criticism and our understanding of narrative ethics. An extensive bibliography follows, compiled in Chicago style, documenting primary sources (the literary works) and secondary sources from the diverse theoretical and critical literature that inform this project.

In summary, the introduction has situated the research problem and objectives of this thesis. Literature is often thought of as a visual medium (print on a page) intended for silent reading, yet this study argues that the sonic dimension of prose is both rich and consequential. Modern and contemporary authors have woven sound into their narratives in innovative ways, and these auditory elements carry ethical and political significance—whether it be giving voice to the voiceless, using musical form to suggest harmony or dissonance in society, or foregrounding noise to critique modern life. By traversing multiple literary periods and theoretical approaches, the chapters that follow will demonstrate that exploring the “auditory ethics” of prose offers fresh perspectives on familiar texts and opens up new conversations about the role of the senses in literature. The ultimate hope is that readers of this thesis will come away with a heightened awareness of how listening, as much as seeing or reading, is integral to the experience of narrative, and with an appreciation for the many ways authors use sound to make meaning and pose moral questions.

\chapter{Literature Review}
\section{Sound, Modernity, and the Literary Soundscape}
The field of \textbf{sound studies} provides a crucial starting point for understanding how literature engages with sound. Sound studies scholars have noted that Western culture has long been dominated by what is called “ocularcentrism,” a prioritization of sight over other senses. \href{https://floridapress.blog/2018/12/04/modernist-soundscapes/#:~:text=These%20writers%20challenged%20ocularcentrism%2C%20the,the%20course%20of%20contemporary%20literature}{(floridapress.blog)}. In contrast, the late nineteenth and early twentieth centuries saw the emergence of new audio technologies and a growing interest in auditory experience, often termed the “sonic modernity” of the era. \href{https://journals.openedition.org/ebc/2322#:~:text=Sam%20Halliday%2C%20Sonic%20Modernity%3A%20Representing,Literature%2C%20Culture%20and%20the%20Arts}{(journals.openedition.org)}. \href{https://journals.openedition.org/ebc/2322#:~:text=1%20Sam%20Halliday%E2%80%99s%20dashingly%20titled,%E2%80%A6%5D%20is%20not%20historically%20specific%E2%80%99}{(journals.openedition.org)}. Canadian composer R. Murray Schafer’s influential work on soundscapes introduced the idea that every environment has its own ensemble of sounds, and he famously urged a rethinking of our “soundscape” to counter the unchecked rise of noise pollution (what he dubbed the “sound sewer” of modern life). \href{https://www.acousticbulletin.com/our-visual-focus-part-2-the-eye-versus-the-ear/#:~:text=Bulletin%20www,that%20willingly%20trades%20its}{(acousticbulletin.com)}. Schafer also treated literary descriptions as valuable records of historical soundscapes, calling authors “reliable ‘earwitnesses’” to how the world once sounded. \href{https://jcls.io/article/id/3583/#:~:text=example%2C%20in%20Schafer%20,9%20%20in%20%2033}{(jcls.io)}. This concept of the writer as an “earwitness” underscores that novelists and poets have actively documented and artistically reshaped the auditory textures of their times, not just its visual details.

A number of studies specifically examine the representation of sound in literature, coalescing into what has been termed \textbf{literary acoustics} or the study of the literary soundscape. Sam Halliday’s \textit{Sonic Modernity} (2013), for example, explores how early twentieth-century British and American literature reacted to modern sound phenomena—from urban noise to new music—arguing that “sound is finally receiving the sort of attention the history of aesthetics has traditionally reserved for the image.” \href{https://journals.openedition.org/ebc/2322#:~:text=live%20modernity%21%E2%80%9D%E2%80%99}{(journals.openedition.org)}. Angela Frattarola’s \textit{Modernist Soundscapes: Auditory Technology and the Novel} (2018) likewise examines how technologies like the phonograph and radio influenced modernist writers, enabling new narrative strategies. Frattarola notes that devices such as headphones, which could “pipe sounds from afar into a listener’s headspace,” inspired modernists to experiment with recording the interior monologues of characters in a stream-of-consciousness style. \href{https://floridapress.blog/2018/12/04/modernist-soundscapes/#:~:text=She%20argues%20that%20the%20common,reproduction%20of%20the%20tape%20recorder}{(floridapress.blog)}. Authors like James Joyce and Virginia Woolf, she argues, effectively attempted to transcribe the mind’s “inner sound,” blurring the boundary between external and internal auditory experience. \href{https://floridapress.blog/2018/12/04/modernist-soundscapes/#:~:text=She%20argues%20that%20the%20common,reproduction%20of%20the%20tape%20recorder}{(floridapress.blog)}. Indeed, Woolf’s celebrated use of onomatopoeia can be seen as an effort to capture “the sounds of the world without mediation” on the page. \href{https://floridapress.blog/2018/12/04/modernist-soundscapes/#:~:text=a%20listener%E2%80%99s%20headspace%2C%20inspired%20modernists,reproduction%20of%20the%20tape%20recorder}{(floridapress.blog)}. In Joyce’s works, especially the “Sirens” episode of \textit{Ulysses}, language is pushed to mimic music and ambient noise; Joyce “made words represent music by playing with or even overcoming certain conventional features of language.” \href{https://bibliolore.org/2022/02/02/joyces-musical-sirens/#:~:text=In%20the%20%E2%80%9CSirens%E2%80%9D%20episode%20of,certain%20conventional%20features%20of%20language}{(bibliolore.org)}. These studies situate literary techniques in a wider context of sonic modernity, suggesting that narrative innovations often paralleled contemporary shifts in how people listened and what they heard in an age of cacophony and mechanically reproduced sound.

Another relevant concept is the idea of \textbf{audionarratology}, which merges narratology (the study of narrative structure) with sound studies. Audionarratology examines how narratives can evoke sound and how the act of listening can be thematized in texts. As one scholar puts it, reading a literary text can trigger “phonological recoding” – an inner voice in the reader’s mind – meaning that even silent reading has an implicit auditory component. \href{https://jcls.io/article/id/3583/#:~:text=recitation%20of%20literary%20texts%2C%20where,of%20reading%20whether%20aloud%20or}{(jcls.io)}. From this perspective, devices like alliteration, assonance, and rhythm are not mere ornaments but integral to how a story communicates, effectively creating a “sound design” that the reader experiences internally. \href{https://jcls.io/article/id/3583/#:~:text=approach%20to%20sound%20in%20literature,31}{(jcls.io)}. Moreover, audionarratologists study how texts represent listening within the story world. For example, characters may be depicted as keenly listening to their environment, or entire plots might revolve around sounds (a mysterious noise, a spoken secret, a radio broadcast). These elements contribute to what has been called the “theatre of the mind” in narrative, wherein described sounds conjure vivid imaginative experiences for the reader. \href{https://jcls.io/article/id/3583/#:~:text=There%20have%20been%20numerous%20recent,33%20Snaith}{(jcls.io)}. Recent collections such as \textit{Literature and Sound} (2020, edited by Anna Snaith) and \textit{Audionarratology} (2016, edited by J. Mildorf and T. Kinzel) bring together such analyses, covering topics from the role of musical structures in narrative to the function of silence in fiction. One finding from this body of work is that certain genres historically place more emphasis on ambient sound description—Gothic fiction, for instance, often foregrounds wind, whispers, and other sounds to create suspense. \href{https://jcls.io/article/id/3583/#:~:text=There%20have%20been%20numerous%20recent,33%20Snaith}{(jcls.io)}. In contrast, realist literature might relegate sound to the background unless it directly advances the plot (such as a character overhearing a crucial conversation). This thesis builds on audionarratological insights by investigating how attentiveness to sound in narrative correlates with ethical themes, such as empathy and otherness.

Perhaps the most directly relevant aspect of sound studies to this thesis is the exploration of \textbf{sound and power}, especially the distinction between sound that is meaningful (like speech or music) and “noise” that is unwanted or disruptive. French theorist Jacques Attali’s oft-cited work \textit{Noise: The Political Economy of Music} (1977, English trans. 1985) contends that noise is not just a sonic phenomenon but a social one: “the mere naming of something as noise is a political action” in that it signifies a threat to the established order. \href{https://musicandsoundstudies.wordpress.com/2014/03/03/noise-pt-2-attali/#:~:text=This%20noise%20%E2%80%94%20the%20voice,As%20Attali}{(musicandsoundstudies.wordpress.com)}. \href{https://musicandsoundstudies.wordpress.com/2014/03/03/noise-pt-2-attali/#:~:text=boundaries%2C%20it%20literally%20,As%20he%20concisely}{(musicandsoundstudies.wordpress.com)}. Attali argues that organized sound (music) has historically been used to impose order—he metaphorically describes music as a ritualized “sacrifice” that creates harmony by excluding dissonance. \href{https://musicandsoundstudies.wordpress.com/2014/03/03/noise-pt-2-attali/#:~:text=To%20explain%20music%20as%20the,audible%20to%20the%20perceptive%20ear}{(musicandsoundstudies.wordpress.com)}. Conversely, noise represents whatever is culturally labeled as disordered or undesirable sound, often the voice of the marginalized or the chaotic din of rebellion. “This noise — the voice of dissent and disorder — threatens to uproot the existing social order,” Attali writes, and therefore societies seek to contain it. \href{https://musicandsoundstudies.wordpress.com/2014/03/03/noise-pt-2-attali/#:~:text=This%20noise%20%E2%80%94%20the%20voice,As%20Attali}{(musicandsoundstudies.wordpress.com)}. His bold claim that “music is prophecy” encapsulates the idea that by listening to what is considered noise, one can hear the rumblings of societal change. \href{https://musicandsoundstudies.wordpress.com/2014/03/03/noise-pt-2-attali/#:~:text=This%20noise%20%E2%80%94%20the%20voice,As%20Attali}{(musicandsoundstudies.wordpress.com)}. In literary terms, we can apply Attali’s insight by asking: which characters or cultural elements are treated as “noise” within a narrative? For example, a novel might portray the songs of a subaltern people as mere background noise in the ears of colonizers, reflecting the latter’s power to dismiss those voices. On the other hand, a writer might stylistically elevate such “noise” into music or poetry on the page, as an act of resistance. Jennifer Stoever’s concept of the “sonic color line” extends the discussion of sound and power into the realm of race. In \textit{The Sonic Color Line: Race and the Cultural Politics of Listening} (2016), Stoever examines how in the United States, listening practices were racialized – how certain voices or sounds were coded as “black” or “white” and valorized or stigmatized accordingly. \href{https://www.aaihs.org/the-sonic-color-line-black-women-and-police-violence/#:~:text=aural%20border%20between%20white%20people,call%20the%20sonic%20color%20line}{(aaihs.org)}. She defines the sonic color line as “the learned cultural mechanism that establishes racial difference through listening habits and uses sound to communicate one’s position vis-à-vis white citizenship.” \href{https://www.aaihs.org/the-sonic-color-line-black-women-and-police-violence/#:~:text=aural%20border%20between%20white%20people,call%20the%20sonic%20color%20line}{(aaihs.org)}. For instance, African American music might be appropriated as entertainment yet the actual voices of black people in public (their cries of protest or grief) might be ignored or labeled as noise. \href{https://www.aaihs.org/the-sonic-color-line-black-women-and-police-violence/#:~:text=presence%20%20of%20a%20long,call%20the%20sonic%20color%20line}{(aaihs.org)}. \href{https://www.aaihs.org/the-sonic-color-line-black-women-and-police-violence/#:~:text=The%20patroller%E2%80%99s%20deliberate%20tone%20ensures,European%20musical%20concepts%20of%20the}{(aaihs.org)}. Stoever’s framework will be particularly relevant when discussing African American literature (Section 4.3), but it also points to a broader methodological approach: in analyzing literary texts, one should pay attention to whose sounds are described as harmonious or meaningful and whose are depicted as cacophonous or negligible. Such narrative choices often mirror societal attitudes and thus carry ethical weight.

In sum, sound studies and related scholarship furnish this thesis with key concepts: the critique of ocularcentrism (and a corresponding validation of auditory ways of knowing). \href{https://floridapress.blog/2018/12/04/modernist-soundscapes/#:~:text=These%20writers%20challenged%20ocularcentrism%2C%20the,the%20course%20of%20contemporary%20literature}{(floridapress.blog)}; the idea of literature as a repository and artistic transformation of historical soundscapes. \href{https://jcls.io/article/id/3583/#:~:text=sonic%20environments%20as%20he%20did,32%20in%20Snaith%202020%2C%2020}{(jcls.io)}; the narrative techniques that evoke sound and require an imagined listening by the reader; and the entanglement of sound with structures of power (who gets to speak, who is “noise”). As we move forward, these ideas will inform readings of specific literary works. Modernist literature’s fascination with sound, for example, can be seen as part of a larger cultural moment that challenged the primacy of visual perception and linear logic, embracing instead the “immediate and unifying” nature of listening. \href{https://floridapress.blog/2018/12/04/modernist-soundscapes/#:~:text=These%20writers%20challenged%20ocularcentrism%2C%20the,the%20course%20of%20contemporary%20literature}{(floridapress.blog)}. Likewise, the attention to dialect and oral tradition in postcolonial writing resonates with the sound-studies recognition that voice and listening are central to identity and agency. The literature review now turns to narrative theory and ethics, and postcolonial theory, to develop those complementary angles.

\section{Narrative Voice and Ethics of Listening}
The field of \textbf{narrative ethics} is concerned with the moral implications of storytelling practices. It asks how narratives position readers in relation to characters and to the truth of the story, and what responsibilities and power dynamics are involved in the act of narration. One foundational idea in narrative ethics (influenced by the philosophy of Emmanuel Levinas and others) is that encountering the “Other” – an entity outside oneself – is an ethical experience. In literature, this encounter often takes the form of a reader meeting a character or narrator through the medium of text. Scholar Adam Zachary Newton, in his book \textit{Narrative Ethics} (1995), posits that narrative itself is an event of ethical significance because it involves an interaction between self and other: author and reader, narrator and character, text and audience. Rather than viewing stories as closed worlds unto themselves, Newton suggests we consider the transactional nature of reading – essentially a form of listening. We “listen” to the voices of narrators and characters, and this act can mirror real-life ethical situations, such as listening to another person’s testimony or plea.

\textbf{Voice} in narrative theory is a multifaceted concept, but centrally it refers to the perspective and agency behind the words on the page. Who is speaking, and from what position? Is the narrative voice omniscient and authoritative, or limited and perhaps unreliable? Is the character allowed to speak in their own voice (through first-person narration or interior monologue), or is their story told by someone else? These questions have ethical dimensions. Wayne C. Booth, in \textit{The Company We Keep: An Ethics of Fiction} (1988), argued that narratives implicitly ask readers to trust certain voices and values; the “company” a reader keeps by believing or sympathizing with a narrator is an ethical choice shaped by the text. If a novel silences a character or presents them only through another’s judgment, it raises ethical concerns about misrepresentation or marginalization within the story’s world. By contrast, a novel that features multiple voices in dialogue (sometimes literally speaking in the text, sometimes through shifts in narrative point of view) can enact a kind of ethical pluralism, acknowledging no single perspective has a monopoly on truth.

Here we can invoke Mikhail Bakhtin’s influential concept of \textbf{polyphony} in the novel. In his study of Dostoevsky, Bakhtin praised the author’s ability to create a “polyphonic” narrative in which characters’ voices are not subordinated to a single authoritative narrator, but rather coexist and interact on relatively autonomous terms. Such a novel is “multi-voiced” and resists reducing characters to mere mouthpieces for the author’s viewpoint. Polyphony in this sense aligns with an ethical stance: it honors the distinctiveness of each voice, much as an ethical society would honor the autonomy and perspective of each individual. Bakhtin’s work was not explicitly ethical in a prescriptive sense, but later scholars have noted the ethical resonance of his ideas: a truly dialogic work is one that models listening and responsiveness among voices.

If we bring these narrative theories into conversation with sound, an intriguing parallel emerges: \textbf{reading as listening}. When an author crafts a first-person narrative or interior monologue, the reader is essentially placed in the position of a listener, overhearing the character’s voice. For instance, when we read the soliloquies in Faulkner’s \textit{The Sound and the Fury} or the interior monologue of Molly Bloom at the end of Joyce’s \textit{Ulysses}, we are asked to “listen in” on very intimate, unfiltered speech. This can create empathy, but it can also be demanding or uncomfortable—such narratives often lack the guiding commentary of an outside narrator, meaning we must make sense of the voice on its own terms. Philosopher and literary critic Martha Nussbaum has argued that this kind of engagement with a character’s inner voice can expand the reader’s moral imagination, training us in patience, attention, and empathy (as she discusses in works like \textit{Love’s Knowledge}, 1990). From the perspective of narrative ethics, the formal choice to grant a character their own voice is an ethical one: it invites the reader to treat that character as a subject with their own worldview, rather than as an object to be spoken about. In the context of this thesis, we will see authors like Toni Morrison and Maxine Hong Kingston using narrative structure to ensure that marginalized characters (African American women, Chinese American immigrants, etc.) have a voice that the reader must confront and engage with directly.

Another aspect of narrative voice is \textbf{reliability} and trust. A narrator who openly muses, hesitates, or self-corrects might be seen as inviting the reader’s judgment—are they telling the truth? do they see the whole picture? Unreliable narrators (such as the unnamed but clearly biased narrator of Ralph Ellison’s \textit{Invisible Man} during the famous “battle royal” scene, or the child narrator in some of Kingston’s tales who doesn’t fully understand what she hears) force readers to listen between the lines and consider what is not being said. This too is an ethical exercise: the reader must not passively accept a singular voice, but rather become an active, and at times skeptical, listener. The politics of sound enter here when we consider why a narrator might be unreliable or limited – often it is because they are constrained by their social context. In Ellison’s novel, for example, the African American narrator’s perspective is limited not by any personal failing but by the fact that society refuses to “see” or properly hear him (hence he is “invisible”). His first-person voice is passionate and vivid, but part of the novel’s strategy is to reveal to the reader how many voices like his go unheard in the real world. Thus, the reader is ethically implicated: having heard his story, what will we do with it? This aligns with what some narrative ethicists describe as the transfer of ethical responsibility to the reader once a story is told.

Within narrative theory, the \textbf{act of listening} itself has been thematized by some scholars. James Phelan and Peter Rabinowitz, in their work on narrative communication, use a model where authors “tell” stories to readers, but readers must also be willing to listen (or read receptively). The ethics of reading, then, involves a willingness to be responsive to a text’s voice. One might draw on the metaphor of an ethical listener: just as in interpersonal ethics, listening to another person with attention and openness is a virtue, similarly, reading a narrative empathetically can be seen as practicing an ethical form of listening. Of course, this can be taken too far—one should not accept everything a text implies uncritically, just as one might question a speaker who might be deceitful or biased. Thus, a balance must be struck between empathetic listening and critical analysis.

In sum, narrative theory contributes to our understanding of auditory ethics by illuminating how \textbf{voice} is constructed in fiction and why it matters. A story can either encourage us to listen to its characters or can drown them out with an overbearing narrative authority. Many of the works this thesis examines are notable for their polyvocality or for elevating voices that literary tradition had often sidelined (like those speaking in dialect or non-standard English). The “politics of sound” here intersects with the politics of narration: whose voice drives the narrative and whose voices are included as part of the story’s fabric. When we later analyze, for example, Zora Neale Hurston’s use of vernacular dialogue or Kingston’s blending of memoir with transcribed oral legends, we will see narrative ethics in action. These writers make deliberate stylistic choices to validate particular ways of speaking and to make readers hear them. In doing so, they implicitly argue for the value of those voices in the moral imagination of their audience.

\section{Postcolonial Perspectives: Voice, Orality, and Power}
Postcolonial theory offers another essential framework for this study, focusing on how literature from formerly colonized or marginalized cultures reclaims voice and agency. A key concern of postcolonial analysis is encapsulated in Gayatri Chakravorty Spivak’s famous question: “Can the subaltern speak?” \href{https://forsea.co/silent-voices-the-subaltern-can-speak-have-always-spoken/#:~:text=%E2%80%9CCan%20the%20subaltern%20speak%3F%E2%80%9D%2C%20postcolonial,and%20always%20will%20speak}{(forsea.co)}. By “subaltern,” Spivak refers to those of inferior rank in colonial and class hierarchies – the colonized, the oppressed, particularly subaltern women – whose voices have been historically silenced or filtered through colonial discourse. Spivak’s answer was pessimistic: the subaltern’s voice is often so overwritten by dominant narratives that, in the strictest sense, it cannot be heard authentically by the centers of power. This critique has galvanized postcolonial writers and scholars to find ways to let the subaltern speak, or to highlight the structural reasons why that speech is not heard.

One of the most direct ways literature engages with this issue is through \textbf{orality} – integrating oral traditions, spoken language patterns, and storytelling techniques from non-Western or indigenous cultures into written texts. In many colonized societies, oral literature (folk tales, epics, songs, proverbs) was the primary medium of cultural transmission. Colonial powers often disparaged orality as a sign of backwardness, elevating the written word (especially when written in the colonizer’s language) as the only legitimate culture. Thus, when postcolonial or anticolonial writers incorporate oral elements into novels or memoirs, they are performing a double act: artistically enriching their narratives and asserting the worth of their heritage. There is an ethical